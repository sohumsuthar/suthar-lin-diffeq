\documentclass{report}
\usepackage{fancyhdr}
\usepackage{amsmath}
\input{preamble}
\input{macros}
\input{letterfonts}
\usepackage{listings}
\usepackage{hyperref}



\pagestyle{fancy}
\fancyhf{} 
\rhead{\thepage} 
%\lhead{\hyperlink{https://www.sohumsuthar.com}{\underline{Sohum Suthar}}}

\newcommand{\mylink}[2]{\underline{\href{#1}{#2}}}

\lhead{\mylink{https://www.sohumsuthar.com}{Sohum Suthar}}

\fancyhead[C]{MATH 2568 Homework \#2}

\title{\Huge{MATH 2568}\\\huge{Dr. Krishnan}\\homework \#2}
\author{\Huge{\mylink{https://www.sohumsuthar.com}{Sohum Suthar}}}
\date{\huge {01/28/2024}}

\begin{document}

\maketitle
\newpage
\pdfbookmark[section]{\contentsname}{toc}	
\tableofcontents
\pagebreak





\section*{2.3: Exercise 1} 
\addcontentsline{toc}{section}{2.3: Exercise 1} 

In Exercise 1 determine
whether the given matrix is in reduced echelon form.

\begin{enumerate}
    \item[(1)] 
    \[
\begin{bmatrix} 1 & -1 & 0 & 1 \\0 &  1 &  0 &  -6\\ 0 & 0 & 1 & 0 \end{bmatrix}
\]
    
\end{enumerate}

\sol

\begin{enumerate}
\item[(1)] The matrix is not in reduced echelon form because the fourth column, although containing a leading 1, has a non-zero entry below it in the second row (-6). Reduced echelon form requires all entries below leading 1's to be zero. Therefore, the matrix is $\boxed{\text{not}}$ in reduced echelon form.
\end{enumerate}










\section*{2.3: Exercise 8} 
\addcontentsline{toc}{section}{2.3: Exercise 8} 

\begin{enumerate}
    \item[(a)] Consider the $2\times 2$ matrix
    \[ 
   \begin{bmatrix} a & b \\ c & 1 \end{bmatrix}
    \]
    where $a,b,c\in\mathbb{R}$ and $a\neq 0$.  Show that the above matrix is row equivalent to the following matrix
\[
\begin{bmatrix} 1 & \frac{b}{a} \\ 0 & \frac{a-bc}{a}
\end{bmatrix} 
\]

    
    \item[(b)] Show that the first matrix is the row equivalent to the
identity matrix if and only if $a\neq bc$

\end{enumerate}


\sol

\begin{enumerate}
\item[(a)] We perform row operations to transform the given matrix into the desired form:

\begin{enumerate}
\item Multiply the first row by $\frac{1}{a}$:
\[
\begin{bmatrix} 1 & \frac{b}{a} \\ c & 1 \end{bmatrix}
\]
\item Add $-c$ times the first row to the second row:
\[
\begin{bmatrix} 1 & \frac{b}{a} \\ 0 & \frac{a-bc}{a} \end{bmatrix}
\]
\end{enumerate}


\item[(b)] We continue row operations on the matrix obtained in part (a):

\begin{enumerate}
\item Multiply the second row by $\frac{a}{a-bc}$ (assuming $a \neq bc$):
\[
\begin{bmatrix} 1 & \frac{b}{a} \\ 0 & 1 \end{bmatrix}
\]
\item Add $-\frac{b}{a}$ times the second row to the first row to produce:
\[
\begin{bmatrix} 1 & 0 \\ 0 & 1 \end{bmatrix}
\]
\end{enumerate}


If $a = bc$, the second row operation in part (b) is not valid, as we would be dividing by zero. Therefore, the first matrix is row equivalent to the identity matrix if and only if $a \neq bc$.
\end{enumerate}

\section*{2.3: Exercise 21 (MATLAB)} 
\addcontentsline{toc}{section}{2.3: Exercise 21 (MATLAB)} 

In this Exercise, use elementary row
operations and MATLAB to put each of the given matrices into row echelon
form.  Suppose that the matrix is the augmented matrix for a system of
linear equations.  Is the system consistent or inconsistent?

\begin{enumerate}
    \item[(21)]
\[
\begin{bmatrix}
 2 &  1  &  1   \\
 4 &  2  &  3
\end{bmatrix}.
\]
    
\end{enumerate}

\sol 

\begin{enumerate}
\item[(21)] $\begin{bmatrix}1&0.5000&0.5000 \\ 0&0&1 \end{bmatrix}$ \\
\\
The system is inconsistent due to the fact that the $2^{nd}$ row of the matrix is the equivalent of $0 = 1$
\end{enumerate}

\begin{lstlisting}[language=Matlab, caption={MATLAB code}, xleftmargin=\parindent]
M = [2 1 1; 4 2 3];

M(1, :) = M(1, :) / 2;
M(2,:) = M(2, :) - 4 * M(1, :);


ans = M

>>>2.3ex21
ans =

    1.0000    0.5000    0.5000
         0         0    1.0000

\end{lstlisting}


\section*{2.4: Exercise 3} 
\addcontentsline{toc}{section}{2.4: Exercise 3} 

How many solutions does the equation
\[
A \begin{bmatrix} x_1 \\ x_2 \\ x_3 \end{bmatrix} = \begin{bmatrix} 2 \\ 1 \\ 2 \end{bmatrix}
\]
have for the following choices of $A$.  Explain your reasoning.

\begin{enumerate}
    \item[(a)] \(A = \begin{bmatrix} 1 & 0 & 1 \\ 0 & 1 & 0 \\ 0 & 0 & 0 \end{bmatrix}\)
    \item[(b)] \(A = \begin{bmatrix} 1 & 3 & 1 \\ 2 & 1 & 0 \\ 0 & 0 & 1 \end{bmatrix}\)
    \item[(c)] \(A = \begin{bmatrix} 1 & 1 & 1 \\ 1 & 2 & 1 \\ 1 & 1 & 1 \end{bmatrix}\)
\end{enumerate}

\sol

\begin{enumerate}
\item[(a)] $A = \begin{bmatrix} 1 & 0 & 1 \\ 0 & 1 & 0 \\ 0 & 0 & 0 \end{bmatrix}$

Row three implies $0 = 1$. Therefore, there are $\boxed{\text{no possible solutions}}$ making the system inconsistent.

\item[(b)] $A = \begin{bmatrix} 1 & 3 & 1 \\ 2 & 1 & 0 \\ 0 & 0 & 1 \end{bmatrix}$

The matrix is a row equivalent to the identity matrix $I_3$. Thus, there is $\boxed{\text{one}}$ solution.

\item[(c)] $A = \begin{bmatrix} 1 & 1 & 1 \\ 1 & 2 & 1 \\ 1 & 1 & 1 \end{bmatrix}$

When the matrix is conjugated into echelon form, the rank of $A$ is equal to the rank of the augumented matrix of $2$, implying that there are $\boxed{\infty}$ solutions.	
\end{enumerate}



\section*{2.4: Exercise 4} 
\addcontentsline{toc}{section}{2.4: Exercise 4}  

The augmented matrix of a consistent system of five equations in seven
unknowns has rank equal to three.  How many parameters are needed to
specify all solutions? \\

\sol

\begin{enumerate}
\item[(4)] Since the rank of the augmented matrix is 3, there are 7 - 3 = 4 free variables in the system. Therefore, we need $\boxed{4}$ parameters to specify all solutions.
\end{enumerate}






\section*{2.5: Exercise 4 (MATLAB)} 
\addcontentsline{toc}{section}{2.5: Exercise 4 (MATLAB)} 

In this Exercise use MATLAB to
solve the given system of linear equations to four significant decimal places.

\begin{enumerate}
    \item[(4)]
\[
\begin{array}{rcrcrcr}
     0.1 x_1 & + & \sqrt{5}x_2 & - &   2 x_3 & = & 1 \\
-\sqrt{3}x_1 & + &     \pi x_2 & - & 2.6 x_3 & = & 14.3 \\
         x_1 & - & 7 x_2 & + & \frac{\pi}{2}x_3 & = & \sqrt{2}
\end{array}
\]
    
\end{enumerate}

\sol 

\begin{enumerate}
\item[(4)] $\begin{bmatrix}-7.2216 \\ -1.9048 \\-2.9907 \end{bmatrix}$

\end{enumerate}

\begin{lstlisting}[language=Matlab, caption={MATLAB code}, xleftmargin=\parindent]
L = [0.1, sqrt(5), -2; -sqrt(3), pi, -2.6; 1, -7, pi/2];
R = [1; 14.3; sqrt(2)];

ans = L\R

>> 2.5ex4
ans =

   -7.2216
   -1.9048
   -2.9907

\end{lstlisting}



\section*{2.5: Exercise 11} 
\addcontentsline{toc}{section}{2.5: Exercise 11} 

Let $z=x+iy$ be a complex number. 


\begin{enumerate}
    \item[(a)] Verify that $z\bar{z}=x^2+y^2.$
    \item[(b)] Verify that $\frac{1}{z}=\frac{x-iy}{x^2+y^2}.$
\end{enumerate}

\sol

\begin{enumerate}
\item[(a)] $z\bar{z} = (x + iy)(x - iy) = x^2 - ixy + ixy - i^2y^2 = \boxed{x^2 + y^2}$.

\item[(b)] $\frac{1}{z} = \frac{1}{x + iy} \cdot \frac{x - iy}{x - iy} = \boxed{\frac{x - iy}{x^2 + y^2}}$.
\end{enumerate}





\end{document}
