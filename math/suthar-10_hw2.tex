\documentclass{report}
\usepackage{fancyhdr}
\usepackage{amsmath}
\input{preamble}
\input{macros}
\input{letterfonts}
\usepackage{listings}
\usepackage{hyperref}



\pagestyle{fancy}
\fancyhf{} 
\rhead{\thepage} 
%\lhead{\hyperlink{https://www.sohumsuthar.com}{\underline{Sohum Suthar}}}

\newcommand{\mylink}[2]{\underline{\href{#1}{#2}}}

\lhead{\mylink{https://www.sohumsuthar.com}{Sohum Suthar}}

\fancyhead[C]{MATH 2568 Homework \#2}

\title{\Huge{MATH 2568}\\\huge{Dr. Krishnan}\\homework \#1}
\author{\Huge{\mylink{https://www.sohumsuthar.com}{Sohum Suthar}}}
\date{\huge {01/20/2024}}

\begin{document}

\maketitle
\newpage
\pdfbookmark[section]{\contentsname}{toc}	
\tableofcontents
\pagebreak





\section*{2.3: Exercise 1} 
\addcontentsline{toc}{section}{2.3: Exercise 1} 

In Exercise 1 determine
whether the given matrix is in reduced echelon form.

\begin{enumerate}
    \item[(1)] 
    \[
    \left(\begin{array}{rrrr} 1 & -1 & 0 & 1 \\0 &  1 &  0 &  -6\\ 0 & 0 & 1 & 0 \end{array}\right)
\]
    
\end{enumerate}

\sol 

\begin{enumerate}
\item[(1)]     


%SOL HERE


\end{enumerate}







\section*{2.3: Exercise 8} 
\addcontentsline{toc}{section}{2.3: Exercise 8} 

\begin{enumerate}
    \item[(a)] Consider the $2\times 2$ matrix
    \[ 
    \left(\begin{array}{rrrr} a & b \\ c & 1 \end{array}\right)
    \]
    where $a,b,c\in\mathbb{R}$ and $a\neq 0$.  Show that the above matrix is row equivalent to the following matrix
\[
\left(\begin{array}{cc} 1 & \frac{b}{a} \\ 0 & \frac{a-bc}{a}
\end{array} \right).
\]

    
    \item[(b)] Show that the first matrix is the row equivalent to the
identity matrix if and only if $a\neq bc$

\end{enumerate}

\sol 

\begin{enumerate}
\item[(a)]     

\item[(b)]     

\end{enumerate}

\section*{2.3: Exercise 21} 
\addcontentsline{toc}{section}{2.3: Exercise 21} 

In Exercise 3, let 
$x=(1.2,1.4,-2.45)$ and $y=(-2.6,1.1,0.65)$. Use MATLAB to compute the 
given expression.

\begin{enumerate}
    \item[(3) ]$3.27x-7.4y$
    
\end{enumerate}

\sol 

\begin{enumerate}
\item[(3)]     
$3.27x - 7.4y =  \boxed{(23.1640, -3.5620, -12.8215)}$
\end{enumerate}

\begin{lstlisting}[language=Matlab, caption={MATLAB code}, xleftmargin=\parindent]
x = [1.2 1.4 -2.45];
y = [-2.6 1.1 0.65];

result = 3.27 * x -7.4 * y 

>> ex3hw1

result =

   23.1640   -3.5620  -12.8215

\end{lstlisting}


\section*{1.3: Exercise 18} 
\addcontentsline{toc}{section}{1.3: Exercise 18} 

Every diagonal matrix is a scalar multiple of the identity matrix.

\begin{enumerate}
    \item[(18)] True or False
    
\end{enumerate}

\sol 

\begin{enumerate}
\item[(18)]     
$ \boxed{False}$ \\
Consider the following diagonal matrix:

$
\begin{bmatrix}
2 & 0 \\
0 & 3
\end{bmatrix}
$

This matrix is not a scalar multiple of the identity matrix because there is no scalar that can be multiplied by the identity matrix to produce this matrix.
\end{enumerate}


\section*{1.4: Exercise 9} 
\addcontentsline{toc}{section}{1.4: Exercise 9} 

Find a real number $a$ so that the vectors
\[
x = (1,3,2) \text{ and } y = (2,a,-6)
\]
are perpendicular.\\


\sol 

\begin{enumerate}
\item[(9)]     
Dot product:
\begin{align*}
x \cdot y &= (1,3,2) \cdot (2,a,-6) \\
&= 1 \cdot 2 + 3 \cdot a + 2 \cdot (-6) \\
&= 2 + 3a - 12 \\
&= 3a - 10
\end{align*}

Setting the dot product to zero:
\begin{align*}
3a - 10 &= 0 \\
3a &= 10 \\
a &= \boxed{\frac{10}{3}}
\end{align*}
\end{enumerate}





\section*{2.1: Exercise 7} 
\addcontentsline{toc}{section}{2.1: Exercise 7} 

Find a quadratic polynomial $p(x) = ax^2 + bx + c$
  satisfying $p(0) = 1$, $p(1) = 5$, and $p(-1) = -5$.

\sol 

\begin{enumerate}

\item[(a)]  
\begin{align*}
& \text{Using the given information, we set up a system of equations:} \\
1. & \quad p(0) = a(0)^2 + b(0) + c = 1, \quad \text{which gives } c = 1. \\
2. & \quad p(1) = a(1)^2 + b(1) + 1 = 5, \quad \text{which simplifies to } a + b = 4. \\
3. & \quad p(-1) = a(-1)^2 + b(-1) + 1 = -5, \quad \text{which simplifies to } a - b = -6. \\
& \text{Solving the system of equations:} \\
& \text{Adding equations 2 and 3, we obtain } 2a = -2, \quad \text{so } a = -1. \\
& \text{Substituting } a = -1 \text{ into equation 2, we get } -1 + b = 4, \quad \text{so } b = 5. \\
& \text{Therefore, the quadratic polynomial satisfying the given conditions is:} \\
& \boxed{p(x) = -x^2 + 5x + 1}
\end{align*}
\item[(b)]
\begin{align*}
p(0) &= L \implies c = L \\
p(1) &= M \implies a + b + c = M \implies a + b = M - L \\
p(-1) &= N \implies a - b + c = N \implies a - b = N - L
\end{align*}
\begin{align*}
2a &= (a + b) + (a - b) = M - L + N - L = M + N - 2L \\
\implies a &= \frac{M + N - 2L}{2} \\
2b &= (a + b) - (a - b) = M - L - (N - L) = M - N \\
\implies b &= \frac{M - N}{2} \\
\boxed{(a,b,c) = (\frac{M + N - 2L}{2},\frac{M - N}{2},L)}
\end{align*}


\item[(c)]

\begin{align*}
ax_1^2 + bx_1 + c &= A_1 \\
ax_2^2 + bx_2 + c &= A_2 \\
ax_3^2 + bx_3 + c &= A_3
\end{align*}
\begin{align*}
\boxed{
\begin{array}{ccc}
q(x_1) &= A_1 \implies ax_1^2 + bx_1 + c = A_1 \\
q(x_2) &= A_2 \implies ax_2^2 + bx_2 + c = A_2 \\
q(x_3) &= A_3 \implies ax_3^2 + bx_3 + c = A_3
\end{array}
}
\end{align*}

\end{enumerate}




\section*{2.2: Exercise 5} 
\addcontentsline{toc}{section}{2.2: Exercise 5} 

Find the cosine of the angle between the normal vectors to the planes 
\[
2x - 2y + z = 14 \qquad \text{and}\qquad x + y - 2z = -10
\]


\sol 

\begin{enumerate}
\item[(5)]     

\begin{align*}
\cos \theta &= \frac{\begin{pmatrix} 2 \ -2 \ 1 \end{pmatrix} \cdot \begin{pmatrix} 1 \ 1 \ -2 \end{pmatrix}}{\left| \begin{pmatrix} 2 \ -2 \ 1 \end{pmatrix} \right| \left| \begin{pmatrix} 1 \ 1 \ -2 \end{pmatrix} \right|} \
&= \frac{(2)(1) + (-2)(1) + (1)(-2)}{\sqrt{2^2 + (-2)^2 + 1^2} \sqrt{1^2 + 1^2 + (-2)^2}} \
&= \frac{-2}{\sqrt{9} \sqrt{6}} \
&= \boxed{-\frac{2}{3\sqrt{6}}}.
\end{align*}



\end{enumerate}





\end{document}
