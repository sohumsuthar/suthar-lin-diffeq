\documentclass{report}
\usepackage{fancyhdr}
\usepackage{amsmath}
\input{preamble}
\input{macros}
\input{letterfonts}
\usepackage{listings}
\usepackage{hyperref}



\pagestyle{fancy}
\fancyhf{} 
\rhead{\thepage} 
%\lhead{\hyperlink{https://www.sohumsuthar.com}{\underline{Sohum Suthar}}}

\newcommand{\mylink}[2]{\underline{\href{#1}{#2}}}

\lhead{\mylink{https://www.sohumsuthar.com}{Sohum Suthar}}

\fancyhead[C]{MATH 2568 Homework \#4}

\title{\Huge{MATH 2568}\\\huge{Dr. Krishnan}\\homework \#4}
\author{\Huge{\mylink{https://www.sohumsuthar.com}{Sohum Suthar}}}
\date{\huge {02/17/2024}}

\begin{document}

\maketitle
\newpage
\pdfbookmark[section]{\contentsname}{toc}	
\tableofcontents
\pagebreak





\section*{3.7: exercise 2} 
\addcontentsline{toc}{section}{3.7: exercise 2} 

Let $\alpha \neq 0$ be a real number and let $A$ be an invertible
matrix.  Show that the inverse of the matrix $\alpha A$ is given by
$\frac{1}{\alpha}A^{-1}$.




    


\sol

\begin{enumerate}
\item[(2)]
\begin{equation}
\begin{boxed}{
(\alpha A) \left( \frac{1}{\alpha} A^{-1} \right) = \left( \alpha \cdot \frac{1}{\alpha} \right) (AA^{-1}) = I}
\end{boxed}
\end{equation}
\end{enumerate}










\section*{3.7: exercise 10} 
\addcontentsline{toc}{section}{3.7: exercise 10} For which values of $a,b,c$ is the matrix

\[A = \begin{pmatrix}
1 & a & b \\
0 & 1 & c \\
0 & 0 & 1
\end{pmatrix}
\]

invertible? Find $A^{-1}$ when it exists.




\sol

\begin{enumerate}
\item[(10)]

The matrix $A$ is invertible for any choice of $a$, $b$, and $c$, and
\[
A^{-1} = \begin{pmatrix} 1 & -a & -b + ac \\ 0 & 1 & -c \\ 0 & 0 & 1 \end{pmatrix}.
\]

A matrix is invertible if it is row equivalent to $I_n$. We demonstrate this by performing row reduction on the augmented matrix $(A|I_3)$:

\begin{align*}
\left( \begin{array}{rrr|rrr}
1 & a & b & 1 & 0 & 0 \\
0 & 1 & c & 0 & 1 & 0 \\
0 & 0 & 1 & 0 & 0 & 1
\end{array} \right) &\xrightarrow{R_1 \xrightarrow{-R_1+aR_2}} 
\left( \begin{array}{rrr|rrc}
1 & 0 & 0 & 1 & -a & -b + ac \\
0 & 1 & 0 & 0 & 1 & -c \\
0 & 0 & 1 & 0 & 0 & 1
\end{array} \right)
\end{align*}

Since the resulting matrix is in reduced row echelon form with a leading 1 in each row, $A$ is indeed invertible for any values of $a$, $b$, and $c$. Consequently, the inverse of $A$ is:

$$A^{-1} = \boxed{\begin{pmatrix} 1 & -a & -b + ac \\ 0 & 1 & -c \\ 0 & 0 & 1 \end{pmatrix}}.$$

\end{enumerate}







\section*{3.7: exercise 14} 
\addcontentsline{toc}{section}{3.7: exercise 14} 
Suppose the mapping $L: \mathbb{R}^3 \to \mathbb{R}^2$ is linear and satisfies the following equations:
Let $A$ and $B$ be $3\times 3$ invertible matrices so that

\[A^{-1} = \begin{pmatrix}
1 & 0 & -1 \\
-1 & -1 & 0 \\
0 & 1 & -1
\end{pmatrix}
\quad\text{and}\quad
B^{-1} = \begin{pmatrix}
1 & 1 & 1 \\
1 & 1 & 0 \\
1 & 0 & 0
\end{pmatrix}
\]

Without computing $A$ or $B$, determine the following:


\begin{enumerate}
\item[(a)] The rank(A)
    \item[(b)] The solution to $Bx = $
\[
\begin{pmatrix}
1 \\
1 \\
1
\end{pmatrix}
\]
    \item[(c)] $(2BA)^{-1}$
    \item[(d)] The matrix $C$ so that $ACB+3I_3=0$.
    
\end{enumerate}


\sol

\begin{enumerate}
\item[(a)] is an invertible $3\times 3$ matrix, so $\rank(A)=\boxed{3}$.

\item[(b)] The solution is
\[
x = B^{-1}\begin{pmatrix} 1 \\ 1 \\ 1 \end{pmatrix}= \begin{pmatrix} 1 & 1 & 1 \\ 1 &1 &0 \\ 1 & 0 & 0 \end{pmatrix} \begin{pmatrix} 1 \\ 1 \\ 1 \end{pmatrix}=\boxed{\begin{pmatrix} 3 \\ 2 \\ 1 \end{pmatrix}}
\]

\item[(c)]
\[
(2BA)^{-1}=\frac{1}{2}A^{-1}B^{-1}=\boxed{\frac{1}{2}
\begin{pmatrix} 0 & 1 & 1 \\ -2 &-2 &-1 \\ 0 & 1 & 0 \end{pmatrix}}
\]

\item[(d)] Recall that multiplication on the left by a matrix is not the same as multiplication on the right. We have that
\begin{align*}
ACB=-3I_3\implies&\;\;  A^{-1}ACB=-3A^{-1}I_3 &&\text{multiplying on the left by $A^{-1}$}\\
\implies & \;\; CB=-3A^{-1}\\
\implies & CBB^{-1}=-3A^{-1}B^{-1}&&\text{multiplying on the right by $B^{-1}$}\\
\implies & C=-3A^{-1}B^{-1}\\
\implies & C=\boxed{-3\begin{pmatrix} 0 & 1 & 1 \\ -2 &-2 &-1 \\ 0 & 1 & 0 \end{pmatrix}}
\end{align*}
\end{enumerate}


\section*{3.8: exercise 4} 
\addcontentsline{toc}{section}{3.8: exercise 4} 
Let $A$ be a $2\times 2$ matrix having integer entries.  Find a
condition on the entries of $A$ that guarantees that $A\inv$ has
integer entries.



\sol

\begin{enumerate}
\item[(4)]

The formula for the inverse of a 2x2 matrix, $A^{-1}$, involves a term $\frac{1}{ad-bc}$, where $a$, $b$, $c$, and $d$ are the elements of the matrix $A$. For this inverse to have integer entries (whole numbers), this term needs to be an integer itself.

Since $a$, $b$, $c$, and $d$ are all integers, the only way for $\frac{1}{ad-bc}$ to be an integer is if the absolute value of the expression \boxed{|ad-bc|=1} . In simpler terms, the product of the diagonal elements ($a$ and $d$) minus the product of the off-diagonal elements ($b$ and $c$) must be either 1 or $-1$.


\end{enumerate}



\section*{3.8: exercise 6} 
\addcontentsline{toc}{section}{3.8: exercise 6} 
Suppose a $2\times 2$ matrix $A$ satisfies the following equation:
\begin{equation}
\mathbf{A} \begin{pmatrix} 0 & 2 \\ 1 & 2 \end{pmatrix} = \begin{pmatrix} -1 & 2 \\ 1 & 4 \end{pmatrix}
\end{equation}

Without calculating the entries of $\mathbf{A}$, find $\det(\mathbf{A})$.


\sol

\begin{enumerate}
\item[(6)]
\begin{align*}
\det \left(A \begin{pmatrix} 0 & 2 \\ 1 & 2 \end{pmatrix} \right) &= \det(A) \det \begin{pmatrix} 0 & 2 \\ 1 & 2 \end{pmatrix} \\
&= -2 \det(A) \\
&= \det \begin{pmatrix} -1 & 2 \\ 1 & 4 \end{pmatrix} \\
&= -6.
\end{align*}

Therefore, $\boxed{\det(A) = 3}$.

\end{enumerate}















\end{document}
