\documentclass{report}
\usepackage{fancyhdr}
\usepackage{amsmath}
\input{preamble}
\input{macros}
\input{letterfonts}
\usepackage{listings}
\usepackage{hyperref}



\pagestyle{fancy}
\fancyhf{} 
\rhead{\thepage} 
%\lhead{\hyperlink{https://www.sohumsuthar.com}{\underline{Sohum Suthar}}}

\newcommand{\mylink}[2]{\underline{\href{#1}{#2}}}

\lhead{\mylink{https://www.sohumsuthar.com}{Sohum Suthar}}

\fancyhead[C]{MATH 2568 Homework \#3}

\title{\Huge{MATH 2568}\\\huge{Dr. Krishnan}\\homework \#3}
\author{\Huge{\mylink{https://www.sohumsuthar.com}{Sohum Suthar}}}
\date{\huge {02/10/2024}}

\begin{document}

\maketitle
\newpage
\pdfbookmark[section]{\contentsname}{toc}	
\tableofcontents
\pagebreak





\section*{3.1: exercise 1} 
\addcontentsline{toc}{section}{3.1: exercise 1} 

Let
\begin{equation}
A = \begin{pmatrix} 2 & 1 \\ -1 & 4 \end{pmatrix} \quad \text{and} \quad x = \begin{pmatrix} 3 \\ -2 \end{pmatrix}
\end{equation}


\begin{enumerate}
    \item[(1)] Compute $Ax$





    
\end{enumerate}

\sol

\begin{enumerate}
\item[(1)]
\[
Ax =
\left(\begin{array}{rr} 2 & 1 \\ -1 & 4\end{array}\right)
\left(\begin{array}{r} 3 \\ -2\end{array}\right) =
\left(\begin{array}{r} 6 - 2 \\ -3 - 8\end{array}\right) =
\boxed{\left(\begin{array}{r} 4 \\ -11\end{array}\right)}
\]
\end{enumerate}










\section*{3.1: exercise 4} 
\addcontentsline{toc}{section}{3.1: exercise 4} 
    In exercise 4 decide whether or
not the matrix vector product $Ax$ can be computed; if it can, compute the
product.
\begin{enumerate}
    \item[(4)] 

\begin{equation}
A = \begin{pmatrix} 1 & 2 \\ 0 & -5 \end{pmatrix} \quad \text{and} \quad x = \begin{pmatrix} 2 \\ 2 \\4 \end{pmatrix}
\end{equation}
    

\end{enumerate}


\sol

\begin{enumerate}
\item[(4)] The matrix-vector product $Ax$ \boxed{cannot} be computed. 
This is because the number of columns in the matrix $A$ (2) does not match the number of elements in the vector $x$ (3). For matrix-vector multiplication to be defined, these dimensions must be equal.

\end{enumerate}

\section*{3.2: exercise 9} 
\addcontentsline{toc}{section}{3.2: exercise 9} 

\begin{enumerate}
    \item[(9)] Find a $2\times 2$ matrix that reflects vectors in the $(x,y)$ plane across
the line $x=y$.


\end{enumerate}


\sol

\begin{enumerate}
\item[(9)]
Consider a map, denoted by $L_A$, that reflects vectors across the line $y=x$.  We can represent this transformation as $(x, y) \mapsto (y, x)$, where $(x, y)$ is the original vector and $(y, x)$ is its reflected counterpart.

This reflection can be conveniently described using a linear transformation matrix, denoted by $A$. In this case, the appropriate matrix is:

\[
A = \begin{pmatrix} 0 & 1 \\ 1 & 0 \end{pmatrix}
\]
\end{enumerate}




\section*{3.2: exercise 10} 
\addcontentsline{toc}{section}{3.2: exercise 10} 
Suppose the mapping $L: \mathbb{R}^3 \to \mathbb{R}^2$ is linear and satisfies the following equations:

\begin{align*}
L \begin{pmatrix} 1 \\ 0 \\ 0 \end{pmatrix} &= \begin{pmatrix} 1 \\ 2 \end{pmatrix} \\
L \begin{pmatrix} 0 \\ 1 \\ 1 \end{pmatrix} &= \begin{pmatrix} 2 \\ 0 \end{pmatrix} \\
L \begin{pmatrix} 0 \\ 0 \\ 1 \end{pmatrix} &= \begin{pmatrix} -1 \\ 4 \end{pmatrix}\\
\end{align*}
\begin{enumerate}
    \item[(10)] What is the  $2\times 3$ matrix $A$ such that $L = L_A$?


\end{enumerate}


\sol

\begin{enumerate}
\item[(10)]
Because $(0,1,0) = (0,1,1) - (0,0,1)$,  the second column must be:

\[
L \begin{pmatrix} 0 \\ 1\\ 0 \end{pmatrix} = \left(\begin{array}{c} 2 \\ 0 \end{array}\right) - \left(\begin{array}{r} -1 \\ 4 \end{array}\right) = \begin{pmatrix} 3 \\ -4 \end{pmatrix}
\]
\[
A = \boxed{\begin{pmatrix}
1 & 3 & -1 \\
2 & -4 & 4
\end{pmatrix}}
\]
\end{enumerate}


\section*{3.3: exercise 10} 
\addcontentsline{toc}{section}{3.3: exercise 10} 
Determine which of the following maps are linear maps.  If the map is linear give the matrix associated to the linear map. Explain your reasoning.

\begin{enumerate}
 \item $L_1: \mathbb{R}^2 \to \mathbb{R}^2$ maps $\begin{pmatrix} x \\ y \end{pmatrix}$ to $\begin{pmatrix} x + y + 3 \\ 2y + 1 \end{pmatrix}$.
    \item $L_2: \mathbb{R}^2 \to \mathbb{R}^3$ maps $\begin{pmatrix} x \\ y \end{pmatrix}$ to $\begin{pmatrix} \sin x \\ x + y \\ 2y \end{pmatrix}$.
    \item $L_3: \mathbb{R}^2 \to \mathbb{R}$ maps $\begin{pmatrix} x \\ y \end{pmatrix}$ to $x + y$.


\end{enumerate}


\sol

\begin{enumerate}
\item A linear map sends the origin to the origin. However, $L_1 \begin{pmatrix} 0 \\ 0 \end{pmatrix} = \begin{pmatrix} 3 \\ 1 \end{pmatrix} \neq \begin{pmatrix} 0 \\ 0 \end{pmatrix}$, so $L_1$ is not linear.

\item Linear maps $L$ satisfy $L(cX) = cL(X)$. In this case,
\[
cL_2 \begin{pmatrix} x \\ y \end{pmatrix} = \begin{pmatrix} c\sin x \\ cx + cy \\ 2cy \end{pmatrix} \quad \text{and} \quad
L_2 \left( \begin{pmatrix} cx \\ cy \end{pmatrix} \right) = \begin{pmatrix} \sin (cx) \\ cx + cy \\ 2cy \end{pmatrix}
\]
Since $\sin(cx) \neq c\sin(x)$, $L_2$ is not linear.

\item All matrix mappings are linear. We can write
\[
L_3 \begin{pmatrix} x \\ y \end{pmatrix} = x + y = \begin{pmatrix} 1 & 1 \end{pmatrix} \begin{pmatrix} x \\ y \end{pmatrix}
\]
Therefore, $L_3$ is linear with the $1 \times 2$ matrix $A = \begin{pmatrix} 1 & 1 \end{pmatrix}$.
\end{enumerate}



\section*{3.4: exercise 2} 
\addcontentsline{toc}{section}{3.4: exercise 2} 
Write all solutions to the homogeneous system of linear
equations as the general superposition of three vectors.

\begin{enumerate}
    \item[(2)]
\begin{eqnarray*}
x_1+2x_2+x_4-x_5 = 0\\
x_3-2x_4+x_5 = 0
\end{eqnarray*}
\end{enumerate}


\sol

\begin{enumerate}
\item[(2)]








Given the vectors:

\begin{align*}
\mathbf{v}_1 &= \begin{pmatrix} -2 \\ 1 \\ 0 \\ 0 \\ 0 \end{pmatrix}, \\
\mathbf{v}_2 &= \begin{pmatrix} -1 \\ 0 \\ 2 \\ 1 \\ 0 \end{pmatrix}, \\
\mathbf{v}_3 &= \begin{pmatrix} 1 \\ 0 \\ -1 \\ 0 \\ 1 \end{pmatrix}.
\end{align*}

We want to write the matrix of the corresponding homogeneous system and find its solutions.


The homogeneous system consists of equations where each vector $\mathbf{v}_i$ forms a separate row. Therefore, the matrix of the system is:

\[
\mathbf{A} = \begin{pmatrix}
1 & 2 & 0 & 1 & -1 \\
0 & 0 & 1 & -2 & 1
\end{pmatrix}.
\]


We cannot simplify this matrix further using row reduction. Therefore, any solution can be written as a linear combination of the original vectors:

\[
\mathbf{x} = x_2 \mathbf{v}_2 + x_4 \mathbf{v}_4 + x_5 \mathbf{v}_5 = \begin{pmatrix} x_5 - x_4 - 2x_2 \\ x_2 \\ -x_5 + 2x_4 \\ x_4 \\ x_5 \end{pmatrix} = x_2 \begin{pmatrix} -2 \\ 1 \\ 0 \\ 0 \\ 0 \end{pmatrix} + x_4 \begin{pmatrix} -1 \\ 0 \\ 2 \\ 1 \\ 0 \end{pmatrix} + x_5 \begin{pmatrix} 1 \\ 0 \\ -1 \\ 0 \\ 1 \end{pmatrix}.
\]

This expresses the solution space as the span of three linearly independent vectors, confirming that the system has infinitely many solutions.

\end{enumerate}




\section*{3.4: exercise 8(a)} 
\addcontentsline{toc}{section}{3.4: exercise 8(a)} 

Let $A$ be a $3\times 3$ matrix.  Suppose 
\begin{align*}
A \begin{pmatrix} -1 \\ 2 \\ 1 \end{pmatrix} &= \begin{pmatrix} 3 \\ 1 \\ 1 \end{pmatrix} \\
A \begin{pmatrix} 0 \\ 4 \\ 0 \end{pmatrix} &= \begin{pmatrix} -2 \\ 0 \\ 1 \end{pmatrix}
\end{align*}
Find a solution to the inhomogeneous system 
\begin{equation*}
Ax = \begin{pmatrix} 1 \\ 1 \\ 2 \end{pmatrix}
\end{equation*}


\sol

\begin{enumerate}
\item[(a)]


\begin{align*}
% Given equations
A \begin{pmatrix} -1 \\ 2 \\ 1 \end{pmatrix} &= \begin{pmatrix} 3 \\ 1 \\ 1 \end{pmatrix} \\
A \begin{pmatrix} 0 \\ 4 \\ 0 \end{pmatrix} &= \begin{pmatrix} -2 \\ 0 \\ 1 \end{pmatrix} \\
% Combined equation
A \begin{pmatrix} -1 & 0 \\ 2 & 4 \\ 1 & 0 \end{pmatrix} &= \begin{pmatrix} 3 & -2 \\ 1 & 0 \\ 1 & 1 \end{pmatrix} \\
% Solution vector as linear combination
\mathbf{x} &= t \begin{pmatrix} -1 \\ 2 \\ 1 \end{pmatrix} + s \begin{pmatrix} 0 \\ 4 \\ 0 \end{pmatrix} = \begin{pmatrix} -t \\ 2t+4s \\ t \end{pmatrix} \\
% Substitute and solve for scalars
A \begin{pmatrix} -t \\ 2t+4s \\ t \end{pmatrix} &= \begin{pmatrix} 3 & -2 \\ 1 & 0 \\ 1 & 1 \end{pmatrix} \\
\Rightarrow \begin{pmatrix} -at & 2at+4as & at \end{pmatrix} &= \begin{pmatrix} 3 & -2 \\ 1 & 0 \\ 1 & 1 \end{pmatrix} \\
& \Rightarrow t = 1, \ s = -\frac{1}{2} \\
% Final solution
\mathbf{x} &= \begin{pmatrix} -1 \\ 2+4 \cdot (-\frac{1}{2}) \\ 1 \end{pmatrix} \\
&= \begin{pmatrix} -1 \\ 0 \\ 1 \end{pmatrix}
\end{align*}



Therefore, $\mathbf{x} = \boxed{\begin{pmatrix} -1 \\ 0 \\ 1 \end{pmatrix}}$ is a solution.
\end{enumerate}




\section*{3.5: exercise 3} 
\addcontentsline{toc}{section}{3.5: exercise 3} 
In this exercise determine whether or 
not the matrix products $AB$ or $BA$ can be computed for each given pair of 
matrices $A$ and $B$.  If the product is possible, perform the computation.
\begin{enumerate}
\item[(3)]
$A=\left(\begin{array}{rrrr} 8 & 0 & 2 & 3\\ -3 & 0 & -10 &
3\end{array}\right)$
and $B=\left(\begin{array}{rrr} 0 & 2 & 5\\ -1 & 3 & -1 \\ 0 & 1 &
-5\end{array}\right)$.

\end{enumerate}

\sol

\begin{enumerate}
\item[(3)]
The number of columns in A (4) is not equal to the number of rows in B (3), thus AB is not defined.\\
The number of columns in B (3) is not equal to the number of rows in A (2), thus BA is also not defined.\\

Therefore, neither $AB$ nor $BA$ can be computed.

\end{enumerate}












\section*{3.6: exercise 6 (MATLAB)} 
\addcontentsline{toc}{section}{3.6: exercise 6 (MATLAB)} 
In exercise 6 use MATLAB to
verify that $(A+B)C = AC+BC$ for the given matrices.

\begin{enumerate}
    \item[(6)] 
\begin{equation*}
A = \begin{pmatrix} 0 & 2 \\ 2 & 1 \end{pmatrix}, \quad
B = \begin{pmatrix} -2 & 1 \\ 3 & 0 \end{pmatrix}, \quad
C = \begin{pmatrix} 2 & -1 \\ 1 & 5 \end{pmatrix}
\end{equation*}
    
\end{enumerate}

\sol 

\begin{enumerate}
\item[(6)] 
\end{enumerate}

\begin{lstlisting}[language=Matlab, caption={MATLAB code}, xleftmargin=\parindent]
A = [0 2; 2 1];
B = [-2 1; 3 0];
C = [2 -1; 1 5];

leftSide = (A + B) * C

rightSide = A * C + B * C



>>>3.6ex6

leftSide =

    -1    17
    11     0


rightSide =

    -1    17
    11     0

\end{lstlisting}






\section*{3.6: exercise 10 (MATLAB)} 
\addcontentsline{toc}{section}{3.6: exercise 10 (MATLAB)} 

Experimentally, find two symmetric $2\times 2$ matrices $A$ and $B$ for
which the matrix product $AB$ is {\em not\/} symmetric.


\sol 






\begin{lstlisting}[language=Matlab, caption={MATLAB code}, xleftmargin=\parindent]
A = randi(10, 2);
B = A';


C = A * B;
isSymmetric = isequal(C, C');
while issymmetric
    A = randi(10, 2);
    B = A'
    C = A * B;
    isSymmetric = isequal(C, C');
end
fprintf('Non-symmetric product:\n');
A
B
C


>> 3.6ex10
Non-symmetric product C:

A =

     2     2
     3    10


B =

     2     3
     2    10


C =

     8    26
    26   109





\end{lstlisting}







\end{document}
