\documentclass{report}
\usepackage{fancyhdr}
\usepackage{amsmath}
%%%%%%%%%%%%%%%%%%%%%%%%%%%%%%%%%
% PACKAGE IMPORTS
%%%%%%%%%%%%%%%%%%%%%%%%%%%%%%%%%


\usepackage[tmargin=2cm,rmargin=1in,lmargin=1in,margin=0.85in,bmargin=2cm,footskip=.2in]{geometry}
\usepackage{amsmath,amsfonts,amsthm,amssymb,mathtools}
\usepackage[varbb]{newpxmath}
\usepackage{xfrac}
\usepackage[makeroom]{cancel}
\usepackage{mathtools}
\usepackage{bookmark}
\usepackage{enumitem}
\usepackage{hyperref,theoremref}
\hypersetup{
	pdftitle={Assignment},
	colorlinks=true, linkcolor=doc!90,
	bookmarksnumbered=true,
	bookmarksopen=true
}
\usepackage[most,many,breakable]{tcolorbox}
\usepackage{xcolor}
\usepackage{varwidth}
\usepackage{varwidth}
\usepackage{etoolbox}
%\usepackage{authblk}
\usepackage{nameref}
\usepackage{multicol,array}
\usepackage{tikz-cd}
\usepackage[ruled,vlined,linesnumbered]{algorithm2e}
\usepackage{comment} % enables the use of multi-line comments (\ifx \fi) 
\usepackage{import}
\usepackage{xifthen}
\usepackage{pdfpages}
\usepackage{transparent}

\newcommand\mycommfont[1]{\footnotesize\ttfamily\textcolor{blue}{#1}}
\SetCommentSty{mycommfont}
\newcommand{\incfig}[1]{%
    \def\svgwidth{\columnwidth}
    \import{./figures/}{#1.pdf_tex}
}

\usepackage{tikzsymbols}
\renewcommand\qedsymbol{$\Laughey$}


%\usepackage{import}
%\usepackage{xifthen}
%\usepackage{pdfpages}
%\usepackage{transparent}


%%%%%%%%%%%%%%%%%%%%%%%%%%%%%%
% SELF MADE COLORS
%%%%%%%%%%%%%%%%%%%%%%%%%%%%%%



\definecolor{myg}{RGB}{56, 140, 70}
\definecolor{myb}{RGB}{45, 111, 177}
\definecolor{myr}{RGB}{199, 68, 64}
\definecolor{mytheorembg}{HTML}{F2F2F9}
\definecolor{mytheoremfr}{HTML}{00007B}
\definecolor{mylenmabg}{HTML}{FFFAF8}
\definecolor{mylenmafr}{HTML}{983b0f}
\definecolor{mypropbg}{HTML}{f2fbfc}
\definecolor{mypropfr}{HTML}{191971}
\definecolor{myexamplebg}{HTML}{F2FBF8}
\definecolor{myexamplefr}{HTML}{88D6D1}
\definecolor{myexampleti}{HTML}{2A7F7F}
\definecolor{mydefinitbg}{HTML}{E5E5FF}
\definecolor{mydefinitfr}{HTML}{3F3FA3}
\definecolor{notesgreen}{RGB}{0,162,0}
\definecolor{myp}{RGB}{197, 92, 212}
\definecolor{mygr}{HTML}{2C3338}
\definecolor{myred}{RGB}{127,0,0}
\definecolor{myyellow}{RGB}{169,121,69}
\definecolor{myexercisebg}{HTML}{F2FBF8}
\definecolor{myexercisefg}{HTML}{88D6D1}


%%%%%%%%%%%%%%%%%%%%%%%%%%%%
% TCOLORBOX SETUPS
%%%%%%%%%%%%%%%%%%%%%%%%%%%%

\setlength{\parindent}{1cm}
%================================
% THEOREM BOX
%================================

\tcbuselibrary{theorems,skins,hooks}
\newtcbtheorem[number within=section]{Theorem}{Theorem}
{%
	enhanced,
	breakable,
	colback = mytheorembg,
	frame hidden,
	boxrule = 0sp,
	borderline west = {2pt}{0pt}{mytheoremfr},
	sharp corners,
	detach title,
	before upper = \tcbtitle\par\smallskip,
	coltitle = mytheoremfr,
	fonttitle = \bfseries\sffamily,
	description font = \mdseries,
	separator sign none,
	segmentation style={solid, mytheoremfr},
}
{th}

\tcbuselibrary{theorems,skins,hooks}
\newtcbtheorem[number within=chapter]{theorem}{Theorem}
{%
	enhanced,
	breakable,
	colback = mytheorembg,
	frame hidden,
	boxrule = 0sp,
	borderline west = {2pt}{0pt}{mytheoremfr},
	sharp corners,
	detach title,
	before upper = \tcbtitle\par\smallskip,
	coltitle = mytheoremfr,
	fonttitle = \bfseries\sffamily,
	description font = \mdseries,
	separator sign none,
	segmentation style={solid, mytheoremfr},
}
{th}


\tcbuselibrary{theorems,skins,hooks}
\newtcolorbox{Theoremcon}
{%
	enhanced
	,breakable
	,colback = mytheorembg
	,frame hidden
	,boxrule = 0sp
	,borderline west = {2pt}{0pt}{mytheoremfr}
	,sharp corners
	,description font = \mdseries
	,separator sign none
}

%================================
% Corollery
%================================
\tcbuselibrary{theorems,skins,hooks}
\newtcbtheorem[number within=section]{Corollary}{Corollary}
{%
	enhanced
	,breakable
	,colback = myp!10
	,frame hidden
	,boxrule = 0sp
	,borderline west = {2pt}{0pt}{myp!85!black}
	,sharp corners
	,detach title
	,before upper = \tcbtitle\par\smallskip
	,coltitle = myp!85!black
	,fonttitle = \bfseries\sffamily
	,description font = \mdseries
	,separator sign none
	,segmentation style={solid, myp!85!black}
}
{th}
\tcbuselibrary{theorems,skins,hooks}
\newtcbtheorem[number within=chapter]{corollary}{Corollary}
{%
	enhanced
	,breakable
	,colback = myp!10
	,frame hidden
	,boxrule = 0sp
	,borderline west = {2pt}{0pt}{myp!85!black}
	,sharp corners
	,detach title
	,before upper = \tcbtitle\par\smallskip
	,coltitle = myp!85!black
	,fonttitle = \bfseries\sffamily
	,description font = \mdseries
	,separator sign none
	,segmentation style={solid, myp!85!black}
}
{th}


%================================
% LENMA
%================================

\tcbuselibrary{theorems,skins,hooks}
\newtcbtheorem[number within=section]{Lenma}{Lenma}
{%
	enhanced,
	breakable,
	colback = mylenmabg,
	frame hidden,
	boxrule = 0sp,
	borderline west = {2pt}{0pt}{mylenmafr},
	sharp corners,
	detach title,
	before upper = \tcbtitle\par\smallskip,
	coltitle = mylenmafr,
	fonttitle = \bfseries\sffamily,
	description font = \mdseries,
	separator sign none,
	segmentation style={solid, mylenmafr},
}
{th}

\tcbuselibrary{theorems,skins,hooks}
\newtcbtheorem[number within=chapter]{lenma}{Lenma}
{%
	enhanced,
	breakable,
	colback = mylenmabg,
	frame hidden,
	boxrule = 0sp,
	borderline west = {2pt}{0pt}{mylenmafr},
	sharp corners,
	detach title,
	before upper = \tcbtitle\par\smallskip,
	coltitle = mylenmafr,
	fonttitle = \bfseries\sffamily,
	description font = \mdseries,
	separator sign none,
	segmentation style={solid, mylenmafr},
}
{th}


%================================
% PROPOSITION
%================================

\tcbuselibrary{theorems,skins,hooks}
\newtcbtheorem[number within=section]{Prop}{Proposition}
{%
	enhanced,
	breakable,
	colback = mypropbg,
	frame hidden,
	boxrule = 0sp,
	borderline west = {2pt}{0pt}{mypropfr},
	sharp corners,
	detach title,
	before upper = \tcbtitle\par\smallskip,
	coltitle = mypropfr,
	fonttitle = \bfseries\sffamily,
	description font = \mdseries,
	separator sign none,
	segmentation style={solid, mypropfr},
}
{th}

\tcbuselibrary{theorems,skins,hooks}
\newtcbtheorem[number within=chapter]{prop}{Proposition}
{%
	enhanced,
	breakable,
	colback = mypropbg,
	frame hidden,
	boxrule = 0sp,
	borderline west = {2pt}{0pt}{mypropfr},
	sharp corners,
	detach title,
	before upper = \tcbtitle\par\smallskip,
	coltitle = mypropfr,
	fonttitle = \bfseries\sffamily,
	description font = \mdseries,
	separator sign none,
	segmentation style={solid, mypropfr},
}
{th}


%================================
% CLAIM
%================================

\tcbuselibrary{theorems,skins,hooks}
\newtcbtheorem[number within=section]{claim}{Claim}
{%
	enhanced
	,breakable
	,colback = myg!10
	,frame hidden
	,boxrule = 0sp
	,borderline west = {2pt}{0pt}{myg}
	,sharp corners
	,detach title
	,before upper = \tcbtitle\par\smallskip
	,coltitle = myg!85!black
	,fonttitle = \bfseries\sffamily
	,description font = \mdseries
	,separator sign none
	,segmentation style={solid, myg!85!black}
}
{th}



%================================
% Exercise
%================================

\tcbuselibrary{theorems,skins,hooks}
\newtcbtheorem[number within=section]{Exercise}{Exercise}
{%
	enhanced,
	breakable,
	colback = myexercisebg,
	frame hidden,
	boxrule = 0sp,
	borderline west = {2pt}{0pt}{myexercisefg},
	sharp corners,
	detach title,
	before upper = \tcbtitle\par\smallskip,
	coltitle = myexercisefg,
	fonttitle = \bfseries\sffamily,
	description font = \mdseries,
	separator sign none,
	segmentation style={solid, myexercisefg},
}
{th}

\tcbuselibrary{theorems,skins,hooks}
\newtcbtheorem[number within=chapter]{exercise}{Exercise}
{%
	enhanced,
	breakable,
	colback = myexercisebg,
	frame hidden,
	boxrule = 0sp,
	borderline west = {2pt}{0pt}{myexercisefg},
	sharp corners,
	detach title,
	before upper = \tcbtitle\par\smallskip,
	coltitle = myexercisefg,
	fonttitle = \bfseries\sffamily,
	description font = \mdseries,
	separator sign none,
	segmentation style={solid, myexercisefg},
}
{th}

%================================
% EXAMPLE BOX
%================================

\newtcbtheorem[number within=section]{Example}{Example}
{%
	colback = myexamplebg
	,breakable
	,colframe = myexamplefr
	,coltitle = myexampleti
	,boxrule = 1pt
	,sharp corners
	,detach title
	,before upper=\tcbtitle\par\smallskip
	,fonttitle = \bfseries
	,description font = \mdseries
	,separator sign none
	,description delimiters parenthesis
}
{ex}

\newtcbtheorem[number within=chapter]{example}{Example}
{%
	colback = myexamplebg
	,breakable
	,colframe = myexamplefr
	,coltitle = myexampleti
	,boxrule = 1pt
	,sharp corners
	,detach title
	,before upper=\tcbtitle\par\smallskip
	,fonttitle = \bfseries
	,description font = \mdseries
	,separator sign none
	,description delimiters parenthesis
}
{ex}

%================================
% DEFINITION BOX
%================================

\newtcbtheorem[number within=section]{Definition}{Definition}{enhanced,
	before skip=2mm,after skip=2mm, colback=red!5,colframe=red!80!black,boxrule=0.5mm,
	attach boxed title to top left={xshift=1cm,yshift*=1mm-\tcboxedtitleheight}, varwidth boxed title*=-3cm,
	boxed title style={frame code={
					\path[fill=tcbcolback]
					([yshift=-1mm,xshift=-1mm]frame.north west)
					arc[start angle=0,end angle=180,radius=1mm]
					([yshift=-1mm,xshift=1mm]frame.north east)
					arc[start angle=180,end angle=0,radius=1mm];
					\path[left color=tcbcolback!60!black,right color=tcbcolback!60!black,
						middle color=tcbcolback!80!black]
					([xshift=-2mm]frame.north west) -- ([xshift=2mm]frame.north east)
					[rounded corners=1mm]-- ([xshift=1mm,yshift=-1mm]frame.north east)
					-- (frame.south east) -- (frame.south west)
					-- ([xshift=-1mm,yshift=-1mm]frame.north west)
					[sharp corners]-- cycle;
				},interior engine=empty,
		},
	fonttitle=\bfseries,
	title={#2},#1}{def}
\newtcbtheorem[number within=chapter]{definition}{Definition}{enhanced,
	before skip=2mm,after skip=2mm, colback=red!5,colframe=red!80!black,boxrule=0.5mm,
	attach boxed title to top left={xshift=1cm,yshift*=1mm-\tcboxedtitleheight}, varwidth boxed title*=-3cm,
	boxed title style={frame code={
					\path[fill=tcbcolback]
					([yshift=-1mm,xshift=-1mm]frame.north west)
					arc[start angle=0,end angle=180,radius=1mm]
					([yshift=-1mm,xshift=1mm]frame.north east)
					arc[start angle=180,end angle=0,radius=1mm];
					\path[left color=tcbcolback!60!black,right color=tcbcolback!60!black,
						middle color=tcbcolback!80!black]
					([xshift=-2mm]frame.north west) -- ([xshift=2mm]frame.north east)
					[rounded corners=1mm]-- ([xshift=1mm,yshift=-1mm]frame.north east)
					-- (frame.south east) -- (frame.south west)
					-- ([xshift=-1mm,yshift=-1mm]frame.north west)
					[sharp corners]-- cycle;
				},interior engine=empty,
		},
	fonttitle=\bfseries,
	title={#2},#1}{def}



%================================
% Solution BOX
%================================

\makeatletter
\newtcbtheorem{question}{Question}{enhanced,
	breakable,
	colback=white,
	colframe=myb!80!black,
	attach boxed title to top left={yshift*=-\tcboxedtitleheight},
	fonttitle=\bfseries,
	title={#2},
	boxed title size=title,
	boxed title style={%
			sharp corners,
			rounded corners=northwest,
			colback=tcbcolframe,
			boxrule=0pt,
		},
	underlay boxed title={%
			\path[fill=tcbcolframe] (title.south west)--(title.south east)
			to[out=0, in=180] ([xshift=5mm]title.east)--
			(title.center-|frame.east)
			[rounded corners=\kvtcb@arc] |-
			(frame.north) -| cycle;
		},
	#1
}{def}
\makeatother

%================================
% SOLUTION BOX
%================================

\makeatletter
\newtcolorbox{solution}{enhanced,
	breakable,
	colback=white,
	colframe=myg!80!black,
	attach boxed title to top left={yshift*=-\tcboxedtitleheight},
	title=Solution,
	boxed title size=title,
	boxed title style={%
			sharp corners,
			rounded corners=northwest,
			colback=tcbcolframe,
			boxrule=0pt,
		},
	underlay boxed title={%
			\path[fill=tcbcolframe] (title.south west)--(title.south east)
			to[out=0, in=180] ([xshift=5mm]title.east)--
			(title.center-|frame.east)
			[rounded corners=\kvtcb@arc] |-
			(frame.north) -| cycle;
		},
}
\makeatother

%================================
% Question BOX
%================================

\makeatletter
\newtcbtheorem{qstion}{Question}{enhanced,
	breakable,
	colback=white,
	colframe=mygr,
	attach boxed title to top left={yshift*=-\tcboxedtitleheight},
	fonttitle=\bfseries,
	title={#2},
	boxed title size=title,
	boxed title style={%
			sharp corners,
			rounded corners=northwest,
			colback=tcbcolframe,
			boxrule=0pt,
		},
	underlay boxed title={%
			\path[fill=tcbcolframe] (title.south west)--(title.south east)
			to[out=0, in=180] ([xshift=5mm]title.east)--
			(title.center-|frame.east)
			[rounded corners=\kvtcb@arc] |-
			(frame.north) -| cycle;
		},
	#1
}{def}
\makeatother

\newtcbtheorem[number within=chapter]{wconc}{Wrong Concept}{
	breakable,
	enhanced,
	colback=white,
	colframe=myr,
	arc=0pt,
	outer arc=0pt,
	fonttitle=\bfseries\sffamily\large,
	colbacktitle=myr,
	attach boxed title to top left={},
	boxed title style={
			enhanced,
			skin=enhancedfirst jigsaw,
			arc=3pt,
			bottom=0pt,
			interior style={fill=myr}
		},
	#1
}{def}



%================================
% NOTE BOX
%================================

\usetikzlibrary{arrows,calc,shadows.blur}
\tcbuselibrary{skins}
\newtcolorbox{note}[1][]{%
	enhanced jigsaw,
	colback=gray!20!white,%
	colframe=gray!80!black,
	size=small,
	boxrule=1pt,
	title=\textbf{Note:-},
	halign title=flush center,
	coltitle=black,
	breakable,
	drop shadow=black!50!white,
	attach boxed title to top left={xshift=1cm,yshift=-\tcboxedtitleheight/2,yshifttext=-\tcboxedtitleheight/2},
	minipage boxed title=1.5cm,
	boxed title style={%
			colback=white,
			size=fbox,
			boxrule=1pt,
			boxsep=2pt,
			underlay={%
					\coordinate (dotA) at ($(interior.west) + (-0.5pt,0)$);
					\coordinate (dotB) at ($(interior.east) + (0.5pt,0)$);
					\begin{scope}
						\clip (interior.north west) rectangle ([xshift=3ex]interior.east);
						\filldraw [white, blur shadow={shadow opacity=60, shadow yshift=-.75ex}, rounded corners=2pt] (interior.north west) rectangle (interior.south east);
					\end{scope}
					\begin{scope}[gray!80!black]
						\fill (dotA) circle (2pt);
						\fill (dotB) circle (2pt);
					\end{scope}
				},
		},
	#1,
}

%%%%%%%%%%%%%%%%%%%%%%%%%%%%%%
% SELF MADE COMMANDS
%%%%%%%%%%%%%%%%%%%%%%%%%%%%%%


\newcommand{\thm}[2]{\begin{Theorem}{#1}{}#2\end{Theorem}}
\newcommand{\cor}[2]{\begin{Corollary}{#1}{}#2\end{Corollary}}
\newcommand{\mlenma}[2]{\begin{Lenma}{#1}{}#2\end{Lenma}}
\newcommand{\mprop}[2]{\begin{Prop}{#1}{}#2\end{Prop}}
\newcommand{\clm}[3]{\begin{claim}{#1}{#2}#3\end{claim}}
\newcommand{\wc}[2]{\begin{wconc}{#1}{}\setlength{\parindent}{1cm}#2\end{wconc}}
\newcommand{\thmcon}[1]{\begin{Theoremcon}{#1}\end{Theoremcon}}
\newcommand{\ex}[2]{\begin{Example}{#1}{}#2\end{Example}}
\newcommand{\dfn}[2]{\begin{Definition}[colbacktitle=red!75!black]{#1}{}#2\end{Definition}}
\newcommand{\dfnc}[2]{\begin{definition}[colbacktitle=red!75!black]{#1}{}#2\end{definition}}
\newcommand{\qs}[2]{\begin{question}{#1}{}#2\end{question}}
\newcommand{\pf}[2]{\begin{myproof}[#1]#2\end{myproof}}
\newcommand{\nt}[1]{\begin{note}#1\end{note}}

\newcommand*\circled[1]{\tikz[baseline=(char.base)]{
		\node[shape=circle,draw,inner sep=1pt] (char) {#1};}}
\newcommand\getcurrentref[1]{%
	\ifnumequal{\value{#1}}{0}
	{??}
	{\the\value{#1}}%
}
\newcommand{\getCurrentSectionNumber}{\getcurrentref{section}}
\newenvironment{myproof}[1][\proofname]{%
	\proof[\bfseries #1: ]%
}{\endproof}

\newcommand{\mclm}[2]{\begin{myclaim}[#1]#2\end{myclaim}}
\newenvironment{myclaim}[1][\claimname]{\proof[\bfseries #1: ]}{}

\newcounter{mylabelcounter}

\makeatletter
\newcommand{\setword}[2]{%
	\phantomsection
	#1\def\@currentlabel{\unexpanded{#1}}\label{#2}%
}
\makeatother




\tikzset{
	symbol/.style={
			draw=none,
			every to/.append style={
					edge node={node [sloped, allow upside down, auto=false]{$#1$}}}
		}
}


% deliminators
\DeclarePairedDelimiter{\abs}{\lvert}{\rvert}
\DeclarePairedDelimiter{\norm}{\lVert}{\rVert}

\DeclarePairedDelimiter{\ceil}{\lceil}{\rceil}
\DeclarePairedDelimiter{\floor}{\lfloor}{\rfloor}
\DeclarePairedDelimiter{\round}{\lfloor}{\rceil}

\newsavebox\diffdbox
\newcommand{\slantedromand}{{\mathpalette\makesl{d}}}
\newcommand{\makesl}[2]{%
\begingroup
\sbox{\diffdbox}{$\mathsurround=0pt#1\mathrm{#2}$}%
\pdfsave
\pdfsetmatrix{1 0 0.2 1}%
\rlap{\usebox{\diffdbox}}%
\pdfrestore
\hskip\wd\diffdbox
\endgroup
}
\newcommand{\dd}[1][]{\ensuremath{\mathop{}\!\ifstrempty{#1}{%
\slantedromand\@ifnextchar^{\hspace{0.2ex}}{\hspace{0.1ex}}}%
{\slantedromand\hspace{0.2ex}^{#1}}}}
\ProvideDocumentCommand\dv{o m g}{%
  \ensuremath{%
    \IfValueTF{#3}{%
      \IfNoValueTF{#1}{%
        \frac{\dd #2}{\dd #3}%
      }{%
        \frac{\dd^{#1} #2}{\dd #3^{#1}}%
      }%
    }{%
      \IfNoValueTF{#1}{%
        \frac{\dd}{\dd #2}%
      }{%
        \frac{\dd^{#1}}{\dd #2^{#1}}%
      }%
    }%
  }%
}
\providecommand*{\pdv}[3][]{\frac{\partial^{#1}#2}{\partial#3^{#1}}}
%  - others
\DeclareMathOperator{\Lap}{\mathcal{L}}
\DeclareMathOperator{\Var}{Var} % varience
\DeclareMathOperator{\Cov}{Cov} % covarience
\DeclareMathOperator{\E}{E} % expected

% Since the amsthm package isn't loaded

% I prefer the slanted \leq
\let\oldleq\leq % save them in case they're every wanted
\let\oldgeq\geq
\renewcommand{\leq}{\leqslant}
\renewcommand{\geq}{\geqslant}

% % redefine matrix env to allow for alignment, use r as default
% \renewcommand*\env@matrix[1][r]{\hskip -\arraycolsep
%     \let\@ifnextchar\new@ifnextchar
%     \array{*\c@MaxMatrixCols #1}}


%\usepackage{framed}
%\usepackage{titletoc}
%\usepackage{etoolbox}
%\usepackage{lmodern}


%\patchcmd{\tableofcontents}{\contentsname}{\sffamily\contentsname}{}{}

%\renewenvironment{leftbar}
%{\def\FrameCommand{\hspace{6em}%
%		{\color{myyellow}\vrule width 2pt depth 6pt}\hspace{1em}}%
%	\MakeFramed{\parshape 1 0cm \dimexpr\textwidth-6em\relax\FrameRestore}\vskip2pt%
%}
%{\endMakeFramed}

%\titlecontents{chapter}
%[0em]{\vspace*{2\baselineskip}}
%{\parbox{4.5em}{%
%		\hfill\Huge\sffamily\bfseries\color{myred}\thecontentspage}%
%	\vspace*{-2.3\baselineskip}\leftbar\textsc{\small\chaptername~\thecontentslabel}\\\sffamily}
%{}{\endleftbar}
%\titlecontents{section}
%[8.4em]
%{\sffamily\contentslabel{3em}}{}{}
%{\hspace{0.5em}\nobreak\itshape\color{myred}\contentspage}
%\titlecontents{subsection}
%[8.4em]
%{\sffamily\contentslabel{3em}}{}{}  
%{\hspace{0.5em}\nobreak\itshape\color{myred}\contentspage}



%%%%%%%%%%%%%%%%%%%%%%%%%%%%%%%%%%%%%%%%%%%
% TABLE OF CONTENTS
%%%%%%%%%%%%%%%%%%%%%%%%%%%%%%%%%%%%%%%%%%%

\usepackage{tikz}
\definecolor{doc}{RGB}{0,60,110}
\usepackage{titletoc}
\contentsmargin{0cm}
\titlecontents{chapter}[3.7pc]
{\addvspace{30pt}%
	\begin{tikzpicture}[remember picture, overlay]%
		\draw[fill=doc!60,draw=doc!60] (-7,-.1) rectangle (-0.9,.5);%
		\pgftext[left,x=-3.5cm,y=0.2cm]{\color{white}\Large\sc\bfseries Chapter\ \thecontentslabel};%
	\end{tikzpicture}\color{doc!60}\large\sc\bfseries}%
{}
{}
{\;\titlerule\;\large\sc\bfseries Page \thecontentspage
	\begin{tikzpicture}[remember picture, overlay]
		\draw[fill=doc!60,draw=doc!60] (2pt,0) rectangle (4,0.1pt);
	\end{tikzpicture}}%
\titlecontents{section}[3.7pc]
{\addvspace{2pt}}
{\contentslabel[\thecontentslabel]{2pc}}
{}
{\hfill\small \thecontentspage}
[]
\titlecontents*{subsection}[3.7pc]
{\addvspace{-1pt}\small}
{}
{}
{\ --- \small\thecontentspage}
[ \textbullet\ ][]

\makeatletter
\renewcommand{\tableofcontents}{%
	\chapter*{%
	  \vspace*{-20\p@}%
	  \begin{tikzpicture}[remember picture, overlay]%
		  \pgftext[right,x=15cm,y=0.2cm]{\color{doc!60}\Huge\sc\bfseries \contentsname};%
		  \draw[fill=doc!60,draw=doc!60] (13,-.75) rectangle (20,1);%
		  \clip (13,-.75) rectangle (20,1);
		  \pgftext[right,x=15cm,y=0.2cm]{\color{white}\Huge\sc\bfseries \contentsname};%
	  \end{tikzpicture}}%
	\@starttoc{toc}}
\makeatother


%From M275 "Topology" at SJSU
\newcommand{\id}{\mathrm{id}}
\newcommand{\taking}[1]{\xrightarrow{#1}}
\newcommand{\inv}{^{-1}}

%From M170 "Introduction to Graph Theory" at SJSU
\DeclareMathOperator{\diam}{diam}
\DeclareMathOperator{\ord}{ord}
\newcommand{\defeq}{\overset{\mathrm{def}}{=}}

%From the USAMO .tex files
\newcommand{\ts}{\textsuperscript}
\newcommand{\dg}{^\circ}
\newcommand{\ii}{\item}

% % From Math 55 and Math 145 at Harvard
% \newenvironment{subproof}[1][Proof]{%
% \begin{proof}[#1] \renewcommand{\qedsymbol}{$\blacksquare$}}%
% {\end{proof}}

\newcommand{\liff}{\leftrightarrow}
\newcommand{\lthen}{\rightarrow}
\newcommand{\opname}{\operatorname}
\newcommand{\surjto}{\twoheadrightarrow}
\newcommand{\injto}{\hookrightarrow}
\newcommand{\On}{\mathrm{On}} % ordinals
\DeclareMathOperator{\img}{im} % Image
\DeclareMathOperator{\Img}{Im} % Image
\DeclareMathOperator{\coker}{coker} % Cokernel
\DeclareMathOperator{\Coker}{Coker} % Cokernel
\DeclareMathOperator{\Ker}{Ker} % Kernel
\DeclareMathOperator{\rank}{rank}
\DeclareMathOperator{\Spec}{Spec} % spectrum
\DeclareMathOperator{\Tr}{Tr} % trace
\DeclareMathOperator{\pr}{pr} % projection
\DeclareMathOperator{\ext}{ext} % extension
\DeclareMathOperator{\pred}{pred} % predecessor
\DeclareMathOperator{\dom}{dom} % domain
\DeclareMathOperator{\ran}{ran} % range
\DeclareMathOperator{\Hom}{Hom} % homomorphism
\DeclareMathOperator{\Mor}{Mor} % morphisms
\DeclareMathOperator{\End}{End} % endomorphism

\newcommand{\eps}{\epsilon}
\newcommand{\veps}{\varepsilon}
\newcommand{\ol}{\overline}
\newcommand{\ul}{\underline}
\newcommand{\wt}{\widetilde}
\newcommand{\wh}{\widehat}
\newcommand{\vocab}[1]{\textbf{\color{blue} #1}}
\providecommand{\half}{\frac{1}{2}}
\newcommand{\dang}{\measuredangle} %% Directed angle
\newcommand{\ray}[1]{\overrightarrow{#1}}
\newcommand{\seg}[1]{\overline{#1}}
\newcommand{\arc}[1]{\wideparen{#1}}
\DeclareMathOperator{\cis}{cis}
\DeclareMathOperator*{\lcm}{lcm}
\DeclareMathOperator*{\argmin}{arg min}
\DeclareMathOperator*{\argmax}{arg max}
\newcommand{\cycsum}{\sum_{\mathrm{cyc}}}
\newcommand{\symsum}{\sum_{\mathrm{sym}}}
\newcommand{\cycprod}{\prod_{\mathrm{cyc}}}
\newcommand{\symprod}{\prod_{\mathrm{sym}}}
\newcommand{\Qed}{\begin{flushright}\qed\end{flushright}}
\newcommand{\parinn}{\setlength{\parindent}{1cm}}
\newcommand{\parinf}{\setlength{\parindent}{0cm}}
% \newcommand{\norm}{\|\cdot\|}
\newcommand{\inorm}{\norm_{\infty}}
\newcommand{\opensets}{\{V_{\alpha}\}_{\alpha\in I}}
\newcommand{\oset}{V_{\alpha}}
\newcommand{\opset}[1]{V_{\alpha_{#1}}}
\newcommand{\lub}{\text{lub}}
\newcommand{\del}[2]{\frac{\partial #1}{\partial #2}}
\newcommand{\Del}[3]{\frac{\partial^{#1} #2}{\partial^{#1} #3}}
\newcommand{\deld}[2]{\dfrac{\partial #1}{\partial #2}}
\newcommand{\Deld}[3]{\dfrac{\partial^{#1} #2}{\partial^{#1} #3}}
\newcommand{\lm}{\lambda}
\newcommand{\uin}{\mathbin{\rotatebox[origin=c]{90}{$\in$}}}
\newcommand{\usubset}{\mathbin{\rotatebox[origin=c]{90}{$\subset$}}}
\newcommand{\lt}{\left}
\newcommand{\rt}{\right}
\newcommand{\bs}[1]{\boldsymbol{#1}}
\newcommand{\exs}{\exists}
\newcommand{\st}{\strut}
\newcommand{\dps}[1]{\displaystyle{#1}}

\newcommand{\sol}{\setlength{\parindent}{0cm}\textbf{\textit{Solution:}}\setlength{\parindent}{1cm} }
\newcommand{\solve}[1]{\setlength{\parindent}{0cm}\textbf{\textit{Solution: }}\setlength{\parindent}{1cm}#1 \Qed}

% Things Lie
\newcommand{\kb}{\mathfrak b}
\newcommand{\kg}{\mathfrak g}
\newcommand{\kh}{\mathfrak h}
\newcommand{\kn}{\mathfrak n}
\newcommand{\ku}{\mathfrak u}
\newcommand{\kz}{\mathfrak z}
\DeclareMathOperator{\Ext}{Ext} % Ext functor
\DeclareMathOperator{\Tor}{Tor} % Tor functor
\newcommand{\gl}{\opname{\mathfrak{gl}}} % frak gl group
\renewcommand{\sl}{\opname{\mathfrak{sl}}} % frak sl group chktex 6

% More script letters etc.
\newcommand{\SA}{\mathcal A}
\newcommand{\SB}{\mathcal B}
\newcommand{\SC}{\mathcal C}
\newcommand{\SF}{\mathcal F}
\newcommand{\SG}{\mathcal G}
\newcommand{\SH}{\mathcal H}
\newcommand{\OO}{\mathcal O}

\newcommand{\SCA}{\mathscr A}
\newcommand{\SCB}{\mathscr B}
\newcommand{\SCC}{\mathscr C}
\newcommand{\SCD}{\mathscr D}
\newcommand{\SCE}{\mathscr E}
\newcommand{\SCF}{\mathscr F}
\newcommand{\SCG}{\mathscr G}
\newcommand{\SCH}{\mathscr H}

% Mathfrak primes
\newcommand{\km}{\mathfrak m}
\newcommand{\kp}{\mathfrak p}
\newcommand{\kq}{\mathfrak q}

% number sets
\newcommand{\RR}[1][]{\ensuremath{\ifstrempty{#1}{\mathbb{R}}{\mathbb{R}^{#1}}}}
\newcommand{\NN}[1][]{\ensuremath{\ifstrempty{#1}{\mathbb{N}}{\mathbb{N}^{#1}}}}
\newcommand{\ZZ}[1][]{\ensuremath{\ifstrempty{#1}{\mathbb{Z}}{\mathbb{Z}^{#1}}}}
\newcommand{\QQ}[1][]{\ensuremath{\ifstrempty{#1}{\mathbb{Q}}{\mathbb{Q}^{#1}}}}
\newcommand{\CC}[1][]{\ensuremath{\ifstrempty{#1}{\mathbb{C}}{\mathbb{C}^{#1}}}}
\newcommand{\PP}[1][]{\ensuremath{\ifstrempty{#1}{\mathbb{P}}{\mathbb{P}^{#1}}}}
\newcommand{\HH}[1][]{\ensuremath{\ifstrempty{#1}{\mathbb{H}}{\mathbb{H}^{#1}}}}
\newcommand{\FF}[1][]{\ensuremath{\ifstrempty{#1}{\mathbb{F}}{\mathbb{F}^{#1}}}}
% expected value
\newcommand{\EE}{\ensuremath{\mathbb{E}}}
\newcommand{\charin}{\text{ char }}
\DeclareMathOperator{\sign}{sign}
\DeclareMathOperator{\Aut}{Aut}
\DeclareMathOperator{\Inn}{Inn}
\DeclareMathOperator{\Syl}{Syl}
\DeclareMathOperator{\Gal}{Gal}
\DeclareMathOperator{\GL}{GL} % General linear group
\DeclareMathOperator{\SL}{SL} % Special linear group

%---------------------------------------
% BlackBoard Math Fonts :-
%---------------------------------------

%Captital Letters
\newcommand{\bbA}{\mathbb{A}}	\newcommand{\bbB}{\mathbb{B}}
\newcommand{\bbC}{\mathbb{C}}	\newcommand{\bbD}{\mathbb{D}}
\newcommand{\bbE}{\mathbb{E}}	\newcommand{\bbF}{\mathbb{F}}
\newcommand{\bbG}{\mathbb{G}}	\newcommand{\bbH}{\mathbb{H}}
\newcommand{\bbI}{\mathbb{I}}	\newcommand{\bbJ}{\mathbb{J}}
\newcommand{\bbK}{\mathbb{K}}	\newcommand{\bbL}{\mathbb{L}}
\newcommand{\bbM}{\mathbb{M}}	\newcommand{\bbN}{\mathbb{N}}
\newcommand{\bbO}{\mathbb{O}}	\newcommand{\bbP}{\mathbb{P}}
\newcommand{\bbQ}{\mathbb{Q}}	\newcommand{\bbR}{\mathbb{R}}
\newcommand{\bbS}{\mathbb{S}}	\newcommand{\bbT}{\mathbb{T}}
\newcommand{\bbU}{\mathbb{U}}	\newcommand{\bbV}{\mathbb{V}}
\newcommand{\bbW}{\mathbb{W}}	\newcommand{\bbX}{\mathbb{X}}
\newcommand{\bbY}{\mathbb{Y}}	\newcommand{\bbZ}{\mathbb{Z}}

%---------------------------------------
% MathCal Fonts :-
%---------------------------------------

%Captital Letters
\newcommand{\mcA}{\mathcal{A}}	\newcommand{\mcB}{\mathcal{B}}
\newcommand{\mcC}{\mathcal{C}}	\newcommand{\mcD}{\mathcal{D}}
\newcommand{\mcE}{\mathcal{E}}	\newcommand{\mcF}{\mathcal{F}}
\newcommand{\mcG}{\mathcal{G}}	\newcommand{\mcH}{\mathcal{H}}
\newcommand{\mcI}{\mathcal{I}}	\newcommand{\mcJ}{\mathcal{J}}
\newcommand{\mcK}{\mathcal{K}}	\newcommand{\mcL}{\mathcal{L}}
\newcommand{\mcM}{\mathcal{M}}	\newcommand{\mcN}{\mathcal{N}}
\newcommand{\mcO}{\mathcal{O}}	\newcommand{\mcP}{\mathcal{P}}
\newcommand{\mcQ}{\mathcal{Q}}	\newcommand{\mcR}{\mathcal{R}}
\newcommand{\mcS}{\mathcal{S}}	\newcommand{\mcT}{\mathcal{T}}
\newcommand{\mcU}{\mathcal{U}}	\newcommand{\mcV}{\mathcal{V}}
\newcommand{\mcW}{\mathcal{W}}	\newcommand{\mcX}{\mathcal{X}}
\newcommand{\mcY}{\mathcal{Y}}	\newcommand{\mcZ}{\mathcal{Z}}


%---------------------------------------
% Bold Math Fonts :-
%---------------------------------------

%Captital Letters
\newcommand{\bmA}{\boldsymbol{A}}	\newcommand{\bmB}{\boldsymbol{B}}
\newcommand{\bmC}{\boldsymbol{C}}	\newcommand{\bmD}{\boldsymbol{D}}
\newcommand{\bmE}{\boldsymbol{E}}	\newcommand{\bmF}{\boldsymbol{F}}
\newcommand{\bmG}{\boldsymbol{G}}	\newcommand{\bmH}{\boldsymbol{H}}
\newcommand{\bmI}{\boldsymbol{I}}	\newcommand{\bmJ}{\boldsymbol{J}}
\newcommand{\bmK}{\boldsymbol{K}}	\newcommand{\bmL}{\boldsymbol{L}}
\newcommand{\bmM}{\boldsymbol{M}}	\newcommand{\bmN}{\boldsymbol{N}}
\newcommand{\bmO}{\boldsymbol{O}}	\newcommand{\bmP}{\boldsymbol{P}}
\newcommand{\bmQ}{\boldsymbol{Q}}	\newcommand{\bmR}{\boldsymbol{R}}
\newcommand{\bmS}{\boldsymbol{S}}	\newcommand{\bmT}{\boldsymbol{T}}
\newcommand{\bmU}{\boldsymbol{U}}	\newcommand{\bmV}{\boldsymbol{V}}
\newcommand{\bmW}{\boldsymbol{W}}	\newcommand{\bmX}{\boldsymbol{X}}
\newcommand{\bmY}{\boldsymbol{Y}}	\newcommand{\bmZ}{\boldsymbol{Z}}
%Small Letters
\newcommand{\bma}{\boldsymbol{a}}	\newcommand{\bmb}{\boldsymbol{b}}
\newcommand{\bmc}{\boldsymbol{c}}	\newcommand{\bmd}{\boldsymbol{d}}
\newcommand{\bme}{\boldsymbol{e}}	\newcommand{\bmf}{\boldsymbol{f}}
\newcommand{\bmg}{\boldsymbol{g}}	\newcommand{\bmh}{\boldsymbol{h}}
\newcommand{\bmi}{\boldsymbol{i}}	\newcommand{\bmj}{\boldsymbol{j}}
\newcommand{\bmk}{\boldsymbol{k}}	\newcommand{\bml}{\boldsymbol{l}}
\newcommand{\bmm}{\boldsymbol{m}}	\newcommand{\bmn}{\boldsymbol{n}}
\newcommand{\bmo}{\boldsymbol{o}}	\newcommand{\bmp}{\boldsymbol{p}}
\newcommand{\bmq}{\boldsymbol{q}}	\newcommand{\bmr}{\boldsymbol{r}}
\newcommand{\bms}{\boldsymbol{s}}	\newcommand{\bmt}{\boldsymbol{t}}
\newcommand{\bmu}{\boldsymbol{u}}	\newcommand{\bmv}{\boldsymbol{v}}
\newcommand{\bmw}{\boldsymbol{w}}	\newcommand{\bmx}{\boldsymbol{x}}
\newcommand{\bmy}{\boldsymbol{y}}	\newcommand{\bmz}{\boldsymbol{z}}

%---------------------------------------
% Scr Math Fonts :-
%---------------------------------------

\newcommand{\sA}{{\mathscr{A}}}   \newcommand{\sB}{{\mathscr{B}}}
\newcommand{\sC}{{\mathscr{C}}}   \newcommand{\sD}{{\mathscr{D}}}
\newcommand{\sE}{{\mathscr{E}}}   \newcommand{\sF}{{\mathscr{F}}}
\newcommand{\sG}{{\mathscr{G}}}   \newcommand{\sH}{{\mathscr{H}}}
\newcommand{\sI}{{\mathscr{I}}}   \newcommand{\sJ}{{\mathscr{J}}}
\newcommand{\sK}{{\mathscr{K}}}   \newcommand{\sL}{{\mathscr{L}}}
\newcommand{\sM}{{\mathscr{M}}}   \newcommand{\sN}{{\mathscr{N}}}
\newcommand{\sO}{{\mathscr{O}}}   \newcommand{\sP}{{\mathscr{P}}}
\newcommand{\sQ}{{\mathscr{Q}}}   \newcommand{\sR}{{\mathscr{R}}}
\newcommand{\sS}{{\mathscr{S}}}   \newcommand{\sT}{{\mathscr{T}}}
\newcommand{\sU}{{\mathscr{U}}}   \newcommand{\sV}{{\mathscr{V}}}
\newcommand{\sW}{{\mathscr{W}}}   \newcommand{\sX}{{\mathscr{X}}}
\newcommand{\sY}{{\mathscr{Y}}}   \newcommand{\sZ}{{\mathscr{Z}}}


%---------------------------------------
% Math Fraktur Font
%---------------------------------------

%Captital Letters
\newcommand{\mfA}{\mathfrak{A}}	\newcommand{\mfB}{\mathfrak{B}}
\newcommand{\mfC}{\mathfrak{C}}	\newcommand{\mfD}{\mathfrak{D}}
\newcommand{\mfE}{\mathfrak{E}}	\newcommand{\mfF}{\mathfrak{F}}
\newcommand{\mfG}{\mathfrak{G}}	\newcommand{\mfH}{\mathfrak{H}}
\newcommand{\mfI}{\mathfrak{I}}	\newcommand{\mfJ}{\mathfrak{J}}
\newcommand{\mfK}{\mathfrak{K}}	\newcommand{\mfL}{\mathfrak{L}}
\newcommand{\mfM}{\mathfrak{M}}	\newcommand{\mfN}{\mathfrak{N}}
\newcommand{\mfO}{\mathfrak{O}}	\newcommand{\mfP}{\mathfrak{P}}
\newcommand{\mfQ}{\mathfrak{Q}}	\newcommand{\mfR}{\mathfrak{R}}
\newcommand{\mfS}{\mathfrak{S}}	\newcommand{\mfT}{\mathfrak{T}}
\newcommand{\mfU}{\mathfrak{U}}	\newcommand{\mfV}{\mathfrak{V}}
\newcommand{\mfW}{\mathfrak{W}}	\newcommand{\mfX}{\mathfrak{X}}
\newcommand{\mfY}{\mathfrak{Y}}	\newcommand{\mfZ}{\mathfrak{Z}}
%Small Letters
\newcommand{\mfa}{\mathfrak{a}}	\newcommand{\mfb}{\mathfrak{b}}
\newcommand{\mfc}{\mathfrak{c}}	\newcommand{\mfd}{\mathfrak{d}}
\newcommand{\mfe}{\mathfrak{e}}	\newcommand{\mff}{\mathfrak{f}}
\newcommand{\mfg}{\mathfrak{g}}	\newcommand{\mfh}{\mathfrak{h}}
\newcommand{\mfi}{\mathfrak{i}}	\newcommand{\mfj}{\mathfrak{j}}
\newcommand{\mfk}{\mathfrak{k}}	\newcommand{\mfl}{\mathfrak{l}}
\newcommand{\mfm}{\mathfrak{m}}	\newcommand{\mfn}{\mathfrak{n}}
\newcommand{\mfo}{\mathfrak{o}}	\newcommand{\mfp}{\mathfrak{p}}
\newcommand{\mfq}{\mathfrak{q}}	\newcommand{\mfr}{\mathfrak{r}}
\newcommand{\mfs}{\mathfrak{s}}	\newcommand{\mft}{\mathfrak{t}}
\newcommand{\mfu}{\mathfrak{u}}	\newcommand{\mfv}{\mathfrak{v}}
\newcommand{\mfw}{\mathfrak{w}}	\newcommand{\mfx}{\mathfrak{x}}
\newcommand{\mfy}{\mathfrak{y}}	\newcommand{\mfz}{\mathfrak{z}}

\usepackage{listings}
\usepackage{hyperref}



\pagestyle{fancy}
\fancyhf{} 
\rhead{\thepage} 
%\lhead{\hyperlink{https://www.sohumsuthar.com}{\underline{Sohum Suthar}}}

\newcommand{\mylink}[2]{\underline{\href{#1}{#2}}}

\lhead{\mylink{https://www.sohumsuthar.com}{Sohum Suthar}}

\fancyhead[C]{MATH 2568 Homework \#8}

\title{\Huge{MATH 2568}\\\huge{Dr. Krishnan}\\homework \#8}
\author{\Huge{\mylink{https://www.sohumsuthar.com}{Sohum Suthar}}}
\date{\huge {04/14/2024}}

\begin{document}
\maketitle
\newpage
\pdfbookmark[section]{\contentsname}{toc}	
\tableofcontents
\pagebreak





\section*{8.2: exercise 1} 
\addcontentsline{toc}{section}{8.2: exercise 1} 

The $3\times 3$ matrix
\[
A = \left(\begin{array}{ccc}
1 & 2 & 5 \\
2 & -1 & 1 \\
3 & 1 & 6
\end{array}\right)
\]
has rank two. Let $r_1, r_2, r_3$ be the rows of $A$ and
$c_1, c_2, c_3$ be the columns of $A$. Find all scalars $\alpha_1, \alpha_2, \alpha_3$ and
$\beta_1, \beta_2, \beta_3$ such that
\[
\begin{aligned}
\alpha_1 r_1 + \alpha_2 r_2 + \alpha_3 r_3 & = 0, \\
\beta_1 c_1 + \beta_2 c_2 + \beta_3 c_3 & = 0.
\end{aligned}
\]

\sol
We are given the $3 \times 3$ matrix
\[
A = \left(\begin{array}{ccc}
1 & 2 & 5 \\
2 & -1 & 1 \\
3 & 1 & 6
\end{array}\right),
\]
and we know that it has rank two. This implies that there is a linear dependence among its rows $r_1, r_2, r_3$ and columns $c_1, c_2, c_3$. We need to find the scalars $\alpha_1, \alpha_2, \alpha_3$ and $\beta_1, \beta_2, \beta_3$ such that the following equations are satisfied:
\[
\alpha_1 r_1 + \alpha_2 r_2 + \alpha_3 r_3 = 0 \quad \text{and} \quad \beta_1 c_1 + \beta_2 c_2 + \beta_3 c_3 = 0.
\]


First, we find a nontrivial solution for $\alpha$. For the rows of $A$, we form the equation
\[
\begin{bmatrix}
1 & 2 & 5 \\
2 & -1 & 1 \\
3 & 1 & 6
\end{bmatrix}
\begin{bmatrix}
\alpha_1 \\
\alpha_2 \\
\alpha_3
\end{bmatrix} = 0.
\]
By row reducing $A$, we find that the rows are linearly dependent, specifically the relation $-r_1 - r_2 + r_3 = 0$ holds. Thus, setting $\alpha_3 = 1$ yields $\alpha_1 = -1$ and $\alpha_2 = -1$. Therefore, the possible choice for $\alpha$ is 
\[
\alpha = (\alpha_1, \alpha_2, \alpha_3) = \alpha_3(-1, -1, 1).
\]


Similarly, for the columns of $A$, we solve
\[
\begin{bmatrix}
1 & 2 & 3 \\
2 & -1 & 1 \\
5 & 1 & 6
\end{bmatrix}
\begin{bmatrix}
\beta_1 \\
\beta_2 \\
\beta_3
\end{bmatrix} = 0.
\]
Upon row reduction, it is found that the columns follow the relation $-7c_1 - 9c_2 + 5c_3 = 0$. Assigning $\beta_3 = 1$, we determine $\beta_1 = -\frac{7}{5}$ and $\beta_2 = -\frac{9}{5}$. Hence, the scalars for $\beta$ are
\[
\beta = (\beta_1, \beta_2, \beta_3) = \beta_3\left(-\frac{7}{5}, -\frac{9}{5}, 1\right).
\]




\section*{8.2: exercise 6} 
\addcontentsline{toc}{section}{8.2: exercise 6}

Let $B$ be an $m \times p$ matrix and let $C$ be a $p \times n$ matrix. Consider the $m \times n$ matrix $A = BC$. We are asked to prove the following inequality for the rank of matrix $A$:
\[
\rank(A) \leq \min\{\rank(B),\;\rank(C)\}.
\]

\sol

We aim to prove that $\rank(A) \leq \min\{\rank(B),\rank(C)\}$ for matrices $B$ of size $m \times p$ and $C$ of size $p \times n$, where $A = BC$. 


Consider the matrix $A = BC$. We observe that the columns of $A$ are formed as linear combinations of the columns of $B$. Consequently, the column space of $A$ cannot span more than the column space of $B$. Therefore, $\rank(A) \leq \rank(B)$.

Next, we examine the transpose of $A$, denoted as $A^t$. By the properties of matrix transposition, $A^t = C^tB^t$. Similar to the previous argument, the columns of $A^t$ are linear combinations of the columns of $C^t$, implying that the column space of $A^t$ cannot exceed the column space of $C^t$. Thus, $\rank(A^t) \leq \rank(C^t)$.

Since the rank of a matrix remains unchanged under transposition ($\rank(A) = \rank(A^t)$), we conclude that $\rank(A) \leq \min\{\rank(B),\rank(C)\}$.


\section*{8.2: exercise 8} 
\addcontentsline{toc}{section}{8.2: exercise 8}

Consider the $3 \times 3$ matrix
\[
C=\begin{bmatrix} 
1 & 1 & 1 \\ 
-1 & b  &  -1 \\ 
2 & 2 & b^2+1 
\end{bmatrix}.
\]
We need to determine the values of $b$ for which the following conditions are satisfied:

Dimensions of the Range of $C$
\begin{enumerate}
    \item Find all $b$ such that $\dim(\text{range}(C))=3$.
    \item Find all $b$ such that $\dim(\text{range}(C))=2$.
    \item Find all $b$ such that $\dim(\text{range}(C))=1$.
    \item Find all $b$ such that $\dim(\text{range}(C))=0$.
\end{enumerate}

Dimensions of the Null Space of $C$
\begin{enumerate}
    \item Find all $b$ such that $\dim(\text{null space}(C))=3$.
    \item Find all $b$ such that $\dim(\text{null space}(C))=2$.
    \item Find all $b$ such that $\dim(\text{null space}(C))=1$.
    \item Find all $b$ such that $\dim(\text{null space}(C))=0$.
\end{enumerate}

Consistency of a Linear System
Determine all $b$ for which the equation
\begin{equation}
Cx=\begin{bmatrix} 1 \\ 1 \\2 \end{bmatrix}
\end{equation}
is consistent. \textbf{Hint:} Can you convert this into a statement about the range of $C$?\\


\sol

Consider the $3 \times 3$ matrix
\[
C=\begin{bmatrix} 
1 & 1 & 1 \\ 
-1 & b  &  -1 \\ 
2 & 2 & b^2+1 
\end{bmatrix}.
\]


\begin{enumerate}
    \item \textbf{Rank of Matrix $C$:} The dimension of the range of $C$, which is equivalent to the column rank of $C$, equals the row rank of $C$. By applying row reduction techniques to $C$, we find:
    \begin{itemize}
        \item The rank of $C$ is $3$ when $b^2 \neq 1$, as all rows are linearly independent in this case.
        \item The rank of $C$ decreases to $2$ when $b = 1$, indicating one linear dependence among the rows.
        \item The rank of $C$ further reduces to $1$ when $b = -1$, indicating significant linear dependencies among the rows.
    \end{itemize}
    
    \item \textbf{Null Space of $C$:} The dimension of the null space of $C$ is computed as $3 - \text{rank}(C)$. Thus, it varies inversely with the rank of $C$.

\item \textbf{Consistency of a Specific Equation:}
    Equation \eqref{A8.2.1a} is given by
    \[
    Cx = \begin{bmatrix} 1 \\ 1 \\ 2 \end{bmatrix}.
    \]
    This equation has a solution if and only if $\begin{bmatrix} 1 \\ 1 \\ 2 \end{bmatrix}$ is within the range of $C$. The equation is always consistent when $\text{rank}(C) = 3$ (i.e., when $b^2 \neq 1$). Further analysis through row reduction shows:
    \begin{itemize}
        \item The equation is consistent when $b = 1$, as the vector is within the span of the columns of $C$.
        \item The equation is inconsistent when $b = -1$, as the vector lies outside the reduced column span.
    \end{itemize}
\end{enumerate}

\section*{8.3: exercise 5} 
\addcontentsline{toc}{section}{8.3: exercise 5}

Suppose the mapping $L:\mathbb{R}^3 \to \mathbb{R}^2$ is linear and satisfies the following conditions:
\[
L\left(\begin{array}{c}
1 \\
0 \\
0
\end{array}\right) = \left(\begin{array}{c}
1 \\
2
\end{array}\right), \quad
L\left(\begin{array}{c}
0 \\
1 \\
1
\end{array}\right) = \left(\begin{array}{c}
2 \\
0
\end{array}\right), \quad
L\left(\begin{array}{c}
0 \\
0 \\
1
\end{array}\right) = \left(\begin{array}{r}
-1 \\
4
\end{array}\right).
\]
Determine the $2 \times 3$ matrix $A$ such that $L = L_A$, where $L_A$ represents the linear transformation associated with matrix $A$.\\


\sol

Given the linear transformation $L: \mathbb{R}^3 \to \mathbb{R}^2$, we aim to determine the matrix representation $A$ such that $L = L_A$. The transformation properties provided are:
\[
L\left(\begin{array}{c} 1 \\ 0 \\ 0 \end{array}\right) = \left(\begin{array}{c} 1 \\ 2 \end{array}\right) \quad \text{and} \quad L\left(\begin{array}{c} 0 \\ 0 \\ 1 \end{array}\right) = \left(\begin{array}{r} -1 \\ 4 \end{array}\right).
\]

Using linearity and the information given:
\begin{itemize}
    \item The first column of matrix $A$ corresponds to $L(1,0,0)$, giving us the first column of $A$ as $\left(\begin{array}{c} 1 \\ 2 \end{array}\right)$.
    \item The third column of matrix $A$ corresponds to $L(0,0,1)$, giving us the third column of $A$ as $\left(\begin{array}{r} -1 \\ 4 \end{array}\right)$.
\end{itemize}

Furthermore, the transformation of the standard basis vector $(0,1,0)$ can be derived by considering that $(0,1,0) = (0,1,1) - (0,0,1)$. Applying linearity of $L$, we compute:
\[
L \left(\begin{array}{c} 0 \\ 1\\ 0 \end{array}\right) = L\left(\begin{array}{c} 0 \\ 1\\ 1 \end{array}\right) - L\left(\begin{array}{r} 0 \\ 0 \\ 1 \end{array}\right) = \left(\begin{array}{c} 2 \\ 0 \end{array}\right) - \left(\begin{array}{r} -1 \\ 4 \end{array}\right) = \left(\begin{array}{r} 3 \\ -4 \end{array}\right).
\]

This calculation results in the second column of $A$ being $\left(\begin{array}{r} 3 \\ -4 \end{array}\right)$. Consequently, the complete matrix representation of $L$ is:
\[
A = \begin{bmatrix} 1 & 3 & -1 \\ 2 & -4 & 4 \end{bmatrix}.
\]





\section*{8.3: exercise 9} 
\addcontentsline{toc}{section}{8.3: exercise 9}

Given the matrix
\[
A = \begin{pmatrix}
-10 & -6 \\
18 & 11
\end{pmatrix},
\]
determine a basis \(\mathcal{W} = \{w_1, w_2\}\) such that the matrix representation \([L_A]_{\mathcal{W}}\) of the linear map \(L_A\), associated with matrix \(A\), is a diagonal matrix.


\sol

The matrix representation $[L]_{\cal W}$ of the linear transformation $L$ associated with matrix $A$ is diagonal in the basis:
\[
{\cal W} = \left\{\begin{pmatrix} -1 \\ 2 \end{pmatrix}, \begin{pmatrix} 2 \\ -3 \end{pmatrix}\right\}.
\]

The matrix $[L]_{\cal W}$ can be made diagonal if the basis ${\cal W}$ consists of eigenvectors of $L$ corresponding to real eigenvalues. To verify this, we compute the eigenvalues and eigenvectors of $L$:

\begin{itemize}
    \item \textbf{Finding Eigenvalues:} By solving the characteristic polynomial of $A$, which is given by
    \[
    \det(A - \lambda I) = \det\begin{pmatrix}
    -10 - \lambda & -6 \\
    18 & 11 - \lambda
    \end{pmatrix},
    \]
    we find the eigenvalues \(\lambda_1 = 2\) and \(\lambda_2 = -1\).
    
    \item \textbf{Finding Eigenvectors:}
    \begin{enumerate}
        \item For \(\lambda_1 = 2\), solve \( (A - 2I)v = 0 \):
        \[
        \begin{pmatrix}
        -12 & -6 \\
        18 & 9
        \end{pmatrix}
        \begin{pmatrix}
        x \\
        y
        \end{pmatrix} = 0.
        \]
        This yields the eigenvector \( w_1 = (-1, 2)^t \).
        
        \item For \(\lambda_2 = -1\), solve \( (A + I)v = 0 \):
        \[
        \begin{pmatrix}
        -9 & -6 \\
        18 & 12
        \end{pmatrix}
        \begin{pmatrix}
        x \\
        y
        \end{pmatrix} = 0.
        \]
        This yields the eigenvector \( w_2 = (2, -3)^t \).
    \end{enumerate}
\end{itemize}

Thus, the eigenvectors $w_1$ and $w_2$ correspond to the eigenvalues 2 and -1, respectively, and form the basis ${\cal W}$ in which $[L]_{\cal W}$ is diagonal.\\




\section*{9.1: exercise 3} 
\addcontentsline{toc}{section}{9.1: exercise 3}

Determine the point on the line \( y = x \) in \(\mathbb{R}^2\) that has the minimal distance to the point \((1, 6)\). \\


\sol


The line \( y = x \) represents a one-dimensional subspace of \(\mathbb{R}^2\) and can be described by the basis vector \((1, 1)\). To find the point on this line that minimizes the distance to \((1, 6)\), we utilize the projection formula, which states:

\[
\text{Projection of } \mathbf{u} \text{ onto } \mathbf{v} = \frac{\mathbf{u} \cdot \mathbf{v}}{\|\mathbf{v}\|^2} \mathbf{v},
\]
where \(\mathbf{u}\) is the vector from the origin to \((1, 6)\) and \(\mathbf{v}\) is the basis vector \((1, 1)\).

Applying this formula, we compute the dot product and the norm squared of the basis vector:
\[
(1, 1) \cdot (1, 6) = 1 + 6 = 7, \quad \text{and} \quad \|(1, 1)\|^2 = 1^2 + 1^2 = 2.
\]

Thus, the projection of \((1, 6)\) onto the line \(y = x\) is given by:
\[
\frac{7}{2}(1, 1) = \left(\frac{7}{2}, \frac{7}{2}\right).
\]

Therefore, the point \(\left(\frac{7}{2}, \frac{7}{2}\right)\) on the line \(y = x\) is the point that realizes the minimal distance to \((1, 6)\).\\



\section*{9.1: exercise 4} 
\addcontentsline{toc}{section}{9.1: exercise 4}

Determine the vector in the plane defined by the equation \( x - \frac{1}{2} y - \frac{1}{3} z = 0 \) in \(\mathbb{R}^3\) that has the minimal distance to the point \( \mathbf{x}_0 = \begin{pmatrix} 1 \\ 1 \\ 1 \end{pmatrix} \).


\sol



To solve this problem, we first choose basis vectors for the plane. We select:
\[
w_1 = \begin{pmatrix} 1 \\ 2 \\ 0 \end{pmatrix} \quad \text{and} \quad w_2 = \begin{pmatrix} 1 \\ 0 \\ 3 \end{pmatrix}.
\]


The vector \( w \) in the plane that realizes the minimal distance can be expressed as a linear combination of the basis vectors:
\[
w = \alpha_1 w_1 + \alpha_2 w_2.
\]


To find the coefficients \( \alpha_1 \) and \( \alpha_2 \), we define the matrix \( A \) with columns as the basis vectors:
\[
A = \begin{pmatrix} 1 & 1 \\ 2 & 0 \\ 0 & 3 \end{pmatrix}.
\]
Using \( A \), we compute:
\[
A^tA = \begin{pmatrix} 1 & 2 & 0 \\ 1 & 0 & 3 \end{pmatrix} \begin{pmatrix} 1 & 1 \\ 2 & 0 \\ 0 & 3 \end{pmatrix} = \begin{pmatrix} 5 & 1 \\ 1 & 10 \end{pmatrix}.
\]


Applying the formula for the inverse of a \( 2 \times 2 \) matrix:
\[
(A^tA)^{-1} = \begin{pmatrix} 5 & 1 \\ 1 & 10 \end{pmatrix}^{-1} = \frac{1}{5 \times 10 - 1 \times 1} \begin{pmatrix} 10 & -1 \\ -1 & 5 \end{pmatrix} = \frac{1}{49} \begin{pmatrix} 10 & -1 \\ -1 & 5 \end{pmatrix}.
\]


Next, we calculate the product \( A^t x_0 \):
\[
A^t x_0 = \begin{pmatrix} 1 & 2 & 0 \\ 1 & 0 & 3 \end{pmatrix} \begin{pmatrix} 1 \\ 1 \\ 1 \end{pmatrix} = \begin{pmatrix} 3 \\ 4 \end{pmatrix}.
\]
Thus, the coefficients are:
\[
\begin{pmatrix} \alpha_1 \\ \alpha_2 \end{pmatrix} = \frac{1}{49} \begin{pmatrix} 10 & -1 \\ -1 & 5 \end{pmatrix} \begin{pmatrix} 3 \\ 4 \end{pmatrix} = \frac{1}{49} \begin{pmatrix} 26 \\ 17 \end{pmatrix} = \begin{pmatrix} 26/49 \\ 17/49 \end{pmatrix}.
\]

Finally, the vector \( w \) in the plane that is closest to \( \mathbf{x}_0 \) is given by:
\[
w = \frac{26}{49} w_1 + \frac{17}{49} w_2 = \frac{26}{49} \begin{pmatrix} 1 \\ 2 \\ 0 \end{pmatrix} + \frac{17}{49} \begin{pmatrix} 1 \\ 0 \\ 3 \end{pmatrix} = \frac{1}{49} \begin{pmatrix} 43 \\ 52 \\ 51 \end{pmatrix}.
\]\\


\section*{9.1: exercise 5} 
\addcontentsline{toc}{section}{9.1: exercise 5}

Prove that the least squares approximation is unique. Specifically, let \( W \) be a subspace of \( \mathbb{R}^n \) and let \( b \) be a vector in \( \mathbb{R}^n \) outside \( W \). Suppose \( w_1, w_2 \in W \) are two vectors such that the distance \( \|w_1 - b\| = \|w_2 - b\| \) is minimal among all vectors in \( W \). Then demonstrate that \( w_1 = w_2 \).


\sol

To demonstrate the uniqueness of the least squares approximation, consider the following argument:


Assume \( w_1 \) provides the minimal distance to \( b \). By the orthogonality principle, the difference \( w_1 - b \) must be orthogonal to every vector in \( W \). Since \( w_2 - w_1 \) is a vector in \( W \) (being the difference of two vectors in the subspace), it follows that \( w_1 - b \) is also orthogonal to \( w_2 - w_1 \).


Given the orthogonality of \( w_1 - b \) and \( w_2 - w_1 \), we apply the Pythagorean theorem, which states:
\[
\|w_1 - b\|^2 + \|w_2 - w_1\|^2 = \|(w_1 - b) + (w_2 - w_1)\|^2 = \|w_2 - b\|^2.
\]
Since \( \|w_1 - b\| = \|w_2 - b\| \) by assumption, the equation simplifies to:
\[
\|w_1 - b\|^2 + \|w_2 - w_1\|^2 = \|w_1 - b\|^2.
\]


This implies that \( \|w_2 - w_1\|^2 = 0 \), leading to the conclusion that \( w_2 - w_1 = 0 \). Therefore, \( w_1 = w_2 \), establishing the uniqueness of the least squares approximation. \\



\section*{9.1: exercise 13} 
\addcontentsline{toc}{section}{9.1: exercise 13}

Consider the following system of linear equations:
\[
\begin{array}{rcc}
x_1 + x_2 &=& 0, \\
2x_1 - 3x_2 &=& 1, \\
5x_1 - 2x_2 &=& 2.
\end{array}
\]
Verify that the system of linear equations is inconsistent. Then find the least squares approximation solution of the system.


\sol


First, perform elementary row reduction on the augmented matrix of the system:
\[
\left(
\begin{array}{cc|c}
1 & 1 & 0 \\
2 & -3 & 1 \\
5 & -2 & 2 \\
\end{array}
\right) \to \left(
\begin{array}{cc|c}
1 & 1 & 0 \\
0 & -5 & 1 \\
0 & -7 & 2 \\
\end{array}
\right).
\]
This row reduction process reveals that the equations yield inconsistent results, confirming the system's inconsistency.

Since the system is inconsistent, we proceed with the least squares approximation. Consider the coefficient matrix $A$:
\[
A = \begin{pmatrix} 1 & 1 \\ 2 & -3 \\ 5 & -2 \end{pmatrix}.
\]
The matrix $A$ has linearly independent columns, thus we use $A$ itself as $\mathbf{W}$. Let $b = \begin{pmatrix} 0 \\ 1 \\ 2 \end{pmatrix}$ represent the constants from the system.

The least squares solution $\tilde{x}$ is given by:
\[
\tilde{x} = (\mathbf{W}^t\mathbf{W})^{-1}\mathbf{W}^tb.
\]
Calculating $\mathbf{W}^t\mathbf{W}$:
\[
\mathbf{W}^t\mathbf{W} = \begin{pmatrix} 1 & 2 & 5 \\ 1 & -3 & -2 \end{pmatrix} \begin{pmatrix} 1 & 1 \\ 2 & -3 \\ 5 & -2 \end{pmatrix} = \begin{pmatrix} 30 & -15 \\ -15 & 14 \end{pmatrix}.
\]
The inverse of $\mathbf{W}^t\mathbf{W}$:
\[
(\mathbf{W}^t\mathbf{W})^{-1} = \frac{1}{30 \times 14 - (-15) \times (-15)} \begin{pmatrix} 14 & 15 \\ 15 & 30 \end{pmatrix} = \frac{1}{195} \begin{pmatrix} 14 & 15 \\ 15 & 30 \end{pmatrix}.
\]
Then, compute $\mathbf{W}^tb$:
\[
\mathbf{W}^tb = \begin{pmatrix} 1 & 2 & 5 \\ 1 & -3 & -2 \end{pmatrix} \begin{pmatrix} 0 \\ 1 \\ 2 \end{pmatrix} = \begin{pmatrix} 12 \\ -7 \end{pmatrix}.
\]
Thus, the least squares solution is:
\[
\tilde{x} = \frac{1}{195} \begin{pmatrix} 14 & 15 \\ 15 & 30 \end{pmatrix} \begin{pmatrix} 12 \\ -7 \end{pmatrix} = \frac{1}{195} \begin{pmatrix} 63 \\ -30 \end{pmatrix}.
\]
The least squares approximation solution is $\tilde{x} = \frac{1}{195} \begin{pmatrix} 63 \\ -30 \end{pmatrix}$.




\end{document}
